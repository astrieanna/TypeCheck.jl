


% Header, overrides base

    % Make sure that the sphinx doc style knows who it inherits from.
    \def\sphinxdocclass{article}

    % Declare the document class
    \documentclass[letterpaper,10pt,english]{/usr/share/sphinx/texinputs/sphinxhowto}

    % Imports
    \usepackage[utf8]{inputenc}
    \DeclareUnicodeCharacter{00A0}{\\nobreakspace}
    \usepackage[T1]{fontenc}
    \usepackage{babel}
    \usepackage{times}
    \usepackage{import}
    \usepackage[Bjarne]{/usr/share/sphinx/texinputs/fncychap}
    \usepackage{longtable}
    \usepackage{/usr/share/sphinx/texinputs/sphinx}
    \usepackage{multirow}

    \usepackage{amsmath}
    \usepackage{amssymb}
    \usepackage{ucs}
    \usepackage{enumerate}

    % Used to make the Input/Output rules follow around the contents.
    \usepackage{needspace}

    % Pygments requirements
    \usepackage{fancyvrb}
    \usepackage{color}
    % ansi colors additions
    \definecolor{darkgreen}{rgb}{.12,.54,.11}
    \definecolor{lightgray}{gray}{.95}
    \definecolor{brown}{rgb}{0.54,0.27,0.07}
    \definecolor{purple}{rgb}{0.5,0.0,0.5}
    \definecolor{darkgray}{gray}{0.25}
    \definecolor{lightred}{rgb}{1.0,0.39,0.28}
    \definecolor{lightgreen}{rgb}{0.48,0.99,0.0}
    \definecolor{lightblue}{rgb}{0.53,0.81,0.92}
    \definecolor{lightpurple}{rgb}{0.87,0.63,0.87}
    \definecolor{lightcyan}{rgb}{0.5,1.0,0.83}

    % Needed to box output/input
    \usepackage{tikz}
        \usetikzlibrary{calc,arrows,shadows}
    \usepackage[framemethod=tikz]{mdframed}

    \usepackage{alltt}

    % Used to load and display graphics
    \usepackage{graphicx}
    \graphicspath{ {figs/} }
    \usepackage[Export]{adjustbox} % To resize

    % used so that images for notebooks which have spaces in the name can still be included
    \usepackage{grffile}


    % For formatting output while also word wrapping.
    \usepackage{listings}
    \lstset{breaklines=true}
    \lstset{basicstyle=\small\ttfamily}
    \def\smaller{\fontsize{9.5pt}{9.5pt}\selectfont}

    %Pygments definitions
    
\makeatletter
\def\PY@reset{\let\PY@it=\relax \let\PY@bf=\relax%
    \let\PY@ul=\relax \let\PY@tc=\relax%
    \let\PY@bc=\relax \let\PY@ff=\relax}
\def\PY@tok#1{\csname PY@tok@#1\endcsname}
\def\PY@toks#1+{\ifx\relax#1\empty\else%
    \PY@tok{#1}\expandafter\PY@toks\fi}
\def\PY@do#1{\PY@bc{\PY@tc{\PY@ul{%
    \PY@it{\PY@bf{\PY@ff{#1}}}}}}}
\def\PY#1#2{\PY@reset\PY@toks#1+\relax+\PY@do{#2}}

\expandafter\def\csname PY@tok@gd\endcsname{\def\PY@tc##1{\textcolor[rgb]{0.63,0.00,0.00}{##1}}}
\expandafter\def\csname PY@tok@gu\endcsname{\let\PY@bf=\textbf\def\PY@tc##1{\textcolor[rgb]{0.50,0.00,0.50}{##1}}}
\expandafter\def\csname PY@tok@gt\endcsname{\def\PY@tc##1{\textcolor[rgb]{0.00,0.27,0.87}{##1}}}
\expandafter\def\csname PY@tok@gs\endcsname{\let\PY@bf=\textbf}
\expandafter\def\csname PY@tok@gr\endcsname{\def\PY@tc##1{\textcolor[rgb]{1.00,0.00,0.00}{##1}}}
\expandafter\def\csname PY@tok@cm\endcsname{\let\PY@it=\textit\def\PY@tc##1{\textcolor[rgb]{0.25,0.50,0.50}{##1}}}
\expandafter\def\csname PY@tok@vg\endcsname{\def\PY@tc##1{\textcolor[rgb]{0.10,0.09,0.49}{##1}}}
\expandafter\def\csname PY@tok@m\endcsname{\def\PY@tc##1{\textcolor[rgb]{0.40,0.40,0.40}{##1}}}
\expandafter\def\csname PY@tok@mh\endcsname{\def\PY@tc##1{\textcolor[rgb]{0.40,0.40,0.40}{##1}}}
\expandafter\def\csname PY@tok@go\endcsname{\def\PY@tc##1{\textcolor[rgb]{0.53,0.53,0.53}{##1}}}
\expandafter\def\csname PY@tok@ge\endcsname{\let\PY@it=\textit}
\expandafter\def\csname PY@tok@vc\endcsname{\def\PY@tc##1{\textcolor[rgb]{0.10,0.09,0.49}{##1}}}
\expandafter\def\csname PY@tok@il\endcsname{\def\PY@tc##1{\textcolor[rgb]{0.40,0.40,0.40}{##1}}}
\expandafter\def\csname PY@tok@cs\endcsname{\let\PY@it=\textit\def\PY@tc##1{\textcolor[rgb]{0.25,0.50,0.50}{##1}}}
\expandafter\def\csname PY@tok@cp\endcsname{\def\PY@tc##1{\textcolor[rgb]{0.74,0.48,0.00}{##1}}}
\expandafter\def\csname PY@tok@gi\endcsname{\def\PY@tc##1{\textcolor[rgb]{0.00,0.63,0.00}{##1}}}
\expandafter\def\csname PY@tok@gh\endcsname{\let\PY@bf=\textbf\def\PY@tc##1{\textcolor[rgb]{0.00,0.00,0.50}{##1}}}
\expandafter\def\csname PY@tok@ni\endcsname{\let\PY@bf=\textbf\def\PY@tc##1{\textcolor[rgb]{0.60,0.60,0.60}{##1}}}
\expandafter\def\csname PY@tok@nl\endcsname{\def\PY@tc##1{\textcolor[rgb]{0.63,0.63,0.00}{##1}}}
\expandafter\def\csname PY@tok@nn\endcsname{\let\PY@bf=\textbf\def\PY@tc##1{\textcolor[rgb]{0.00,0.00,1.00}{##1}}}
\expandafter\def\csname PY@tok@no\endcsname{\def\PY@tc##1{\textcolor[rgb]{0.53,0.00,0.00}{##1}}}
\expandafter\def\csname PY@tok@na\endcsname{\def\PY@tc##1{\textcolor[rgb]{0.49,0.56,0.16}{##1}}}
\expandafter\def\csname PY@tok@nb\endcsname{\def\PY@tc##1{\textcolor[rgb]{0.00,0.50,0.00}{##1}}}
\expandafter\def\csname PY@tok@nc\endcsname{\let\PY@bf=\textbf\def\PY@tc##1{\textcolor[rgb]{0.00,0.00,1.00}{##1}}}
\expandafter\def\csname PY@tok@nd\endcsname{\def\PY@tc##1{\textcolor[rgb]{0.67,0.13,1.00}{##1}}}
\expandafter\def\csname PY@tok@ne\endcsname{\let\PY@bf=\textbf\def\PY@tc##1{\textcolor[rgb]{0.82,0.25,0.23}{##1}}}
\expandafter\def\csname PY@tok@nf\endcsname{\def\PY@tc##1{\textcolor[rgb]{0.00,0.00,1.00}{##1}}}
\expandafter\def\csname PY@tok@si\endcsname{\let\PY@bf=\textbf\def\PY@tc##1{\textcolor[rgb]{0.73,0.40,0.53}{##1}}}
\expandafter\def\csname PY@tok@s2\endcsname{\def\PY@tc##1{\textcolor[rgb]{0.73,0.13,0.13}{##1}}}
\expandafter\def\csname PY@tok@vi\endcsname{\def\PY@tc##1{\textcolor[rgb]{0.10,0.09,0.49}{##1}}}
\expandafter\def\csname PY@tok@nt\endcsname{\let\PY@bf=\textbf\def\PY@tc##1{\textcolor[rgb]{0.00,0.50,0.00}{##1}}}
\expandafter\def\csname PY@tok@nv\endcsname{\def\PY@tc##1{\textcolor[rgb]{0.10,0.09,0.49}{##1}}}
\expandafter\def\csname PY@tok@s1\endcsname{\def\PY@tc##1{\textcolor[rgb]{0.73,0.13,0.13}{##1}}}
\expandafter\def\csname PY@tok@sh\endcsname{\def\PY@tc##1{\textcolor[rgb]{0.73,0.13,0.13}{##1}}}
\expandafter\def\csname PY@tok@sc\endcsname{\def\PY@tc##1{\textcolor[rgb]{0.73,0.13,0.13}{##1}}}
\expandafter\def\csname PY@tok@sx\endcsname{\def\PY@tc##1{\textcolor[rgb]{0.00,0.50,0.00}{##1}}}
\expandafter\def\csname PY@tok@bp\endcsname{\def\PY@tc##1{\textcolor[rgb]{0.00,0.50,0.00}{##1}}}
\expandafter\def\csname PY@tok@c1\endcsname{\let\PY@it=\textit\def\PY@tc##1{\textcolor[rgb]{0.25,0.50,0.50}{##1}}}
\expandafter\def\csname PY@tok@kc\endcsname{\let\PY@bf=\textbf\def\PY@tc##1{\textcolor[rgb]{0.00,0.50,0.00}{##1}}}
\expandafter\def\csname PY@tok@c\endcsname{\let\PY@it=\textit\def\PY@tc##1{\textcolor[rgb]{0.25,0.50,0.50}{##1}}}
\expandafter\def\csname PY@tok@mf\endcsname{\def\PY@tc##1{\textcolor[rgb]{0.40,0.40,0.40}{##1}}}
\expandafter\def\csname PY@tok@err\endcsname{\def\PY@bc##1{\setlength{\fboxsep}{0pt}\fcolorbox[rgb]{1.00,0.00,0.00}{1,1,1}{\strut ##1}}}
\expandafter\def\csname PY@tok@kd\endcsname{\let\PY@bf=\textbf\def\PY@tc##1{\textcolor[rgb]{0.00,0.50,0.00}{##1}}}
\expandafter\def\csname PY@tok@ss\endcsname{\def\PY@tc##1{\textcolor[rgb]{0.10,0.09,0.49}{##1}}}
\expandafter\def\csname PY@tok@sr\endcsname{\def\PY@tc##1{\textcolor[rgb]{0.73,0.40,0.53}{##1}}}
\expandafter\def\csname PY@tok@mo\endcsname{\def\PY@tc##1{\textcolor[rgb]{0.40,0.40,0.40}{##1}}}
\expandafter\def\csname PY@tok@kn\endcsname{\let\PY@bf=\textbf\def\PY@tc##1{\textcolor[rgb]{0.00,0.50,0.00}{##1}}}
\expandafter\def\csname PY@tok@mi\endcsname{\def\PY@tc##1{\textcolor[rgb]{0.40,0.40,0.40}{##1}}}
\expandafter\def\csname PY@tok@gp\endcsname{\let\PY@bf=\textbf\def\PY@tc##1{\textcolor[rgb]{0.00,0.00,0.50}{##1}}}
\expandafter\def\csname PY@tok@o\endcsname{\def\PY@tc##1{\textcolor[rgb]{0.40,0.40,0.40}{##1}}}
\expandafter\def\csname PY@tok@kr\endcsname{\let\PY@bf=\textbf\def\PY@tc##1{\textcolor[rgb]{0.00,0.50,0.00}{##1}}}
\expandafter\def\csname PY@tok@s\endcsname{\def\PY@tc##1{\textcolor[rgb]{0.73,0.13,0.13}{##1}}}
\expandafter\def\csname PY@tok@kp\endcsname{\def\PY@tc##1{\textcolor[rgb]{0.00,0.50,0.00}{##1}}}
\expandafter\def\csname PY@tok@w\endcsname{\def\PY@tc##1{\textcolor[rgb]{0.73,0.73,0.73}{##1}}}
\expandafter\def\csname PY@tok@kt\endcsname{\def\PY@tc##1{\textcolor[rgb]{0.69,0.00,0.25}{##1}}}
\expandafter\def\csname PY@tok@ow\endcsname{\let\PY@bf=\textbf\def\PY@tc##1{\textcolor[rgb]{0.67,0.13,1.00}{##1}}}
\expandafter\def\csname PY@tok@sb\endcsname{\def\PY@tc##1{\textcolor[rgb]{0.73,0.13,0.13}{##1}}}
\expandafter\def\csname PY@tok@k\endcsname{\let\PY@bf=\textbf\def\PY@tc##1{\textcolor[rgb]{0.00,0.50,0.00}{##1}}}
\expandafter\def\csname PY@tok@se\endcsname{\let\PY@bf=\textbf\def\PY@tc##1{\textcolor[rgb]{0.73,0.40,0.13}{##1}}}
\expandafter\def\csname PY@tok@sd\endcsname{\let\PY@it=\textit\def\PY@tc##1{\textcolor[rgb]{0.73,0.13,0.13}{##1}}}

\def\PYZbs{\char`\\}
\def\PYZus{\char`\_}
\def\PYZob{\char`\{}
\def\PYZcb{\char`\}}
\def\PYZca{\char`\^}
\def\PYZam{\char`\&}
\def\PYZlt{\char`\<}
\def\PYZgt{\char`\>}
\def\PYZsh{\char`\#}
\def\PYZpc{\char`\%}
\def\PYZdl{\char`\$}
\def\PYZhy{\char`\-}
\def\PYZsq{\char`\'}
\def\PYZdq{\char`\"}
\def\PYZti{\char`\~}
% for compatibility with earlier versions
\def\PYZat{@}
\def\PYZlb{[}
\def\PYZrb{]}
\makeatother


    %Set pygments styles if needed...
    
        \definecolor{nbframe-border}{rgb}{0.867,0.867,0.867}
        \definecolor{nbframe-bg}{rgb}{0.969,0.969,0.969}
        \definecolor{nbframe-in-prompt}{rgb}{0.0,0.0,0.502}
        \definecolor{nbframe-out-prompt}{rgb}{0.545,0.0,0.0}

        \newenvironment{ColorVerbatim}
        {\begin{mdframed}[%
            roundcorner=1.0pt, %
            backgroundcolor=nbframe-bg, %
            userdefinedwidth=1\linewidth, %
            leftmargin=0.1\linewidth, %
            innerleftmargin=0pt, %
            innerrightmargin=0pt, %
            linecolor=nbframe-border, %
            linewidth=1pt, %
            usetwoside=false, %
            everyline=true, %
            innerlinewidth=3pt, %
            innerlinecolor=nbframe-bg, %
            middlelinewidth=1pt, %
            middlelinecolor=nbframe-bg, %
            outerlinewidth=0.5pt, %
            outerlinecolor=nbframe-border, %
            needspace=0pt
        ]}
        {\end{mdframed}}
        
        \newenvironment{InvisibleVerbatim}
        {\begin{mdframed}[leftmargin=0.1\linewidth,innerleftmargin=3pt,innerrightmargin=3pt, userdefinedwidth=1\linewidth, linewidth=0pt, linecolor=white, usetwoside=false]}
        {\end{mdframed}}

        \renewenvironment{Verbatim}[1][\unskip]
        {\begin{alltt}\smaller}
        {\end{alltt}}
    

    % Help prevent overflowing lines due to urls and other hard-to-break 
    % entities.  This doesn't catch everything...
    \sloppy

    % Document level variables
    \title{TypeCheck}
    \date{February 19, 2014}
    \release{}
    \author{Unknown Author}
    \renewcommand{\releasename}{}

    % TODO: Add option for the user to specify a logo for his/her export.
    \newcommand{\sphinxlogo}{}

    % Make the index page of the document.
    \makeindex

    % Import sphinx document type specifics.
     


% Body

    % Start of the document
    \begin{document}

        
            \maketitle
        

        


        
        \part{A Summary of Julia}Julia is a new language designed for technical computing. It is as easy
to use and general-purpose as Python, but designed for fast computation,
low-level control, and easy to express math. Julia is high-performance,
dynamically-typed, and JIT-compiled. It is not focused on new ideas, but
on executing existing ideas well, with a focus on being practical and
approachable. As a language, some of Julia's distinctive features
include multiple dispatch, first-class types, and Lisp-style macros.\section{The Julia Type System}Every value in Julia has a type; variables contain values, but do not
themselves have types. Types are arranged into a hierarchy of abstract
and concrete types. Abstract types can have subtypes, but cannot be
instantiated and do not have properties. Concrete types can be
instantiated and have zero or more properties, but cannot have subtypes.
Every type has a super type. At top of the hierarchy is the \texttt{Any}
type; the super type of \texttt{Any} is \texttt{Any}.

In Julia, types are first-class; types are of type DataType, which is
itself of type DataType. Types are used for inference, optimization,
dispatch, and documentation, but not for type checking. Because Julia
still works (but more slowly) without any type inference, all of the
type inference is implemented in the language.

Types can also take parameters; these can be types or \texttt{Int}s. For
example, \texttt{Array\{T,N\}} is parameterized by the element type and
the number of dimensions. Instances of the same type (such as
\texttt{Array}) with different type parameters (say
\texttt{Array\{Int,6\}} and \texttt{Array\{Number,4\}}) are never
subtypes of one another.To define a type, you use the \texttt{type} keyword:

    % Make sure that atleast 4 lines are below the HR
    \needspace{4\baselineskip}

    
        \vspace{6pt}
        \makebox[0.1\linewidth]{\smaller\hfill\tt\color{nbframe-in-prompt}In\hspace{4pt}{[}1{]}:\hspace{4pt}}\\*
        \vspace{-2.65\baselineskip}
        \begin{ColorVerbatim}
            \vspace{-0.7\baselineskip}
            \begin{Verbatim}[commandchars=\\\{\}]
\PY{n+nb}{type} \PY{n}{Point}\PY{p}{\PYZob{}}\PY{n}{T} \PY{o}{\PYZlt{}}\PY{p}{:} \PY{n}{Number}\PY{p}{\PYZcb{}}
    \PY{n}{x}\PY{p}{:}\PY{p}{:}\PY{n}{T}
    \PY{n}{y}\PY{p}{:}\PY{p}{:}\PY{n}{T}
\PY{n}{end}
\end{Verbatim}

            
                \vspace{-0.2\baselineskip}
            
        \end{ColorVerbatim}
    
The Point\{T\} type will have two properties, \texttt{x} and \texttt{y}.
For any \texttt{Point}, the two properties will share a type, and their
type will be a subtype of \texttt{Number}. From the definition, we also
know that when a \texttt{Point} is represented in memory, \texttt{x}
will precede \texttt{y}. Julia types are laid out in memory in a way
compatible with C structs, which made implementing Julia's C calling
functionality easier (and the result more efficient).\section{Introspection in Julia}Julia has admirable introspection and reflection abilities, which are
very useful for writing static analysis. For any named function, you can
get a type-inferred AST with a simple function call:

    % Make sure that atleast 4 lines are below the HR
    \needspace{4\baselineskip}

    
        \vspace{6pt}
        \makebox[0.1\linewidth]{\smaller\hfill\tt\color{nbframe-in-prompt}In\hspace{4pt}{[}2{]}:\hspace{4pt}}\\*
        \vspace{-2.65\baselineskip}
        \begin{ColorVerbatim}
            \vspace{-0.7\baselineskip}
            \begin{Verbatim}[commandchars=\\\{\}]
\PY{n}{function} \PY{n}{foo}\PY{p}{(}\PY{n}{x}\PY{p}{:}\PY{p}{:}\PY{n}{Int}\PY{p}{)}
    \PY{n}{z} \PY{o}{=} \PY{n}{x} \PY{o}{+} \PY{l+m+mi}{5}
    \PY{k}{return} \PY{l+m+mi}{2} \PY{o}{*} \PY{n}{z}
\PY{n}{end}

\PY{n}{code\PYZus{}typed}\PY{p}{(}\PY{n}{foo}\PY{p}{,}\PY{p}{(}\PY{n}{Int}\PY{p}{,}\PY{p}{)}\PY{p}{)}
\end{Verbatim}

            
                \vspace{-0.2\baselineskip}
            
        \end{ColorVerbatim}
    

    

        % If the first block is an image, minipage the image.  Else
        % request a certain amount of space for the input text.
        \needspace{4\baselineskip}
        
        

            % Add document contents.
            
                \makebox[0.1\linewidth]{\smaller\hfill\tt\color{nbframe-out-prompt}Out\hspace{4pt}{[}2{]}:\hspace{4pt}}\\*
                \vspace{-2.55\baselineskip}\begin{InvisibleVerbatim}
                \vspace{-0.5\baselineskip}
\begin{alltt}1-element Array\{Any,1\}:
 :(\$(Expr(:lambda, \{:x\}, \{\{:z\},\{\{:x,Int64,0\},\{:z,Int64,18\}\},\{\}\}, quote
\# In[2], line 2:
        z = top(box)(Int64,top(add\_int)(x::Int64,5))::Int64 \# line 3:
        return top(box)(Int64,top(mul\_int)(2,z::Int64))::Int64
    end)))\end{alltt}

            \end{InvisibleVerbatim}
            
        
    
The \texttt{code\_typed} function takes a function and a method
signature. Every named function is a generic function: it has one or
more methods, each with their own type signature. Julia uses multiple
dispatch, which means that it considers the type, number, and order of
all the arguments to pick the best match to a call.

\texttt{code\_typed} returns an untyped \texttt{Array} of
\texttt{Expr}s, the type that represents a node of the Julia AST. For
many invocations, this \texttt{Array} will only have one element. When
the provided signature could match more than one existing method, all
possible matches are returned. ``Possible'' matches occur when you pass
an abstract type as part of the signature and some methods of the
function accept subtypes of that type. The type of an actual value is
always concrete, so the method that would actually get called would
vary.

    % Make sure that atleast 4 lines are below the HR
    \needspace{4\baselineskip}

    
        \vspace{6pt}
        \makebox[0.1\linewidth]{\smaller\hfill\tt\color{nbframe-in-prompt}In\hspace{4pt}{[}3{]}:\hspace{4pt}}\\*
        \vspace{-2.65\baselineskip}
        \begin{ColorVerbatim}
            \vspace{-0.7\baselineskip}
            \begin{Verbatim}[commandchars=\\\{\}]
\PY{n}{e} \PY{o}{=} \PY{n}{code\PYZus{}typed}\PY{p}{(}\PY{n}{foo}\PY{p}{,}\PY{p}{(}\PY{n}{Int}\PY{p}{,}\PY{p}{)}\PY{p}{)}\PY{p}{[}\PY{l+m+mi}{1}\PY{p}{]} \PY{c}{\PYZsh{}Julia indexes from 1}
\end{Verbatim}

            
                \vspace{-0.2\baselineskip}
            
        \end{ColorVerbatim}
    

    

        % If the first block is an image, minipage the image.  Else
        % request a certain amount of space for the input text.
        \needspace{4\baselineskip}
        
        

            % Add document contents.
            
                \makebox[0.1\linewidth]{\smaller\hfill\tt\color{nbframe-out-prompt}Out\hspace{4pt}{[}3{]}:\hspace{4pt}}\\*
                \vspace{-2.55\baselineskip}\begin{InvisibleVerbatim}
                \vspace{-0.5\baselineskip}
\begin{alltt}:(\$(Expr(:lambda, \{:x\}, \{\{:z\},\{\{:x,Int64,0\},\{:z,Int64,18\}\},\{\}\}, quote
\# In[2], line 2:
        z = top(box)(Int64,top(add\_int)(x::Int64,5))::Int64 \# line 3:
        return top(box)(Int64,top(mul\_int)(2,z::Int64))::Int64
    end)))\end{alltt}

            \end{InvisibleVerbatim}
            
        
    
An \texttt{Expr} has three fields: \texttt{head}, \texttt{args}, and
\texttt{typ}.

    % Make sure that atleast 4 lines are below the HR
    \needspace{4\baselineskip}

    
        \vspace{6pt}
        \makebox[0.1\linewidth]{\smaller\hfill\tt\color{nbframe-in-prompt}In\hspace{4pt}{[}4{]}:\hspace{4pt}}\\*
        \vspace{-2.65\baselineskip}
        \begin{ColorVerbatim}
            \vspace{-0.7\baselineskip}
            \begin{Verbatim}[commandchars=\\\{\}]
\PY{n}{names}\PY{p}{(}\PY{n}{e}\PY{p}{)}
\end{Verbatim}

            
                \vspace{-0.2\baselineskip}
            
        \end{ColorVerbatim}
    

    

        % If the first block is an image, minipage the image.  Else
        % request a certain amount of space for the input text.
        \needspace{4\baselineskip}
        
        

            % Add document contents.
            
                \makebox[0.1\linewidth]{\smaller\hfill\tt\color{nbframe-out-prompt}Out\hspace{4pt}{[}4{]}:\hspace{4pt}}\\*
                \vspace{-2.55\baselineskip}\begin{InvisibleVerbatim}
                \vspace{-0.5\baselineskip}
\begin{alltt}3-element Array\{Symbol,1\}:
 :head
 :args
 :typ\end{alltt}

            \end{InvisibleVerbatim}
            
        
    
\begin{itemize}
\item
  \texttt{head} is a symbol indicating the type of expression. For
  \texttt{Expr}s returned by \texttt{code\_typed}, this will be
  \texttt{:lambda}. (In Julia, \texttt{:foo} is a way to write ``the
  symbol \texttt{foo}''.)
\item
  \texttt{typ} is the return type of the method.
\item
  \texttt{args} is an \texttt{Array} of \texttt{Array}s. It contains
  information about the variables (local, arguments, captured) and body
  of the function. I'll explain more about the structure of
  \texttt{args} as needed in the rest of this document.
\end{itemize}\section{Helper Functions}While introspecting on functions is surprisingly easy, there is a lot of
ugly code created by the unfortunate structure of the \texttt{Expr}
type. As a result, I will start by describing a number of helper
functions, which will give you a better idea of how it is structured.\subsection{A Function to Retrieve Return Types}\texttt{code\_typed} returns \texttt{Expr}s that have lots of type
annotations, including the return type of the function. The outer
\texttt{Expr} with \texttt{head} \texttt{:lambda} will have \texttt{typ}
\texttt{Any}. The third element of \texttt{args} will be another
\texttt{Expr} with \texttt{head} \texttt{:body}. The \texttt{typ}
property of this Expr will be set to the inferred return type of the
function. (There is currently no syntax in Julia to annotate function
return types.)

    % Make sure that atleast 4 lines are below the HR
    \needspace{4\baselineskip}

    
        \vspace{6pt}
        \makebox[0.1\linewidth]{\smaller\hfill\tt\color{nbframe-in-prompt}In\hspace{4pt}{[}5{]}:\hspace{4pt}}\\*
        \vspace{-2.65\baselineskip}
        \begin{ColorVerbatim}
            \vspace{-0.7\baselineskip}
            \begin{Verbatim}[commandchars=\\\{\}]
\PY{n}{code\PYZus{}typed}\PY{p}{(}\PY{n}{foo}\PY{p}{,}\PY{p}{(}\PY{n}{Int}\PY{p}{,}\PY{p}{)}\PY{p}{)}\PY{p}{[}\PY{l+m+mi}{1}\PY{p}{]}\PY{o}{.}\PY{n}{args}\PY{p}{[}\PY{l+m+mi}{3}\PY{p}{]}\PY{o}{.}\PY{n}{typ}
\end{Verbatim}

            
                \vspace{-0.2\baselineskip}
            
        \end{ColorVerbatim}
    

    

        % If the first block is an image, minipage the image.  Else
        % request a certain amount of space for the input text.
        \needspace{4\baselineskip}
        
        

            % Add document contents.
            
                \makebox[0.1\linewidth]{\smaller\hfill\tt\color{nbframe-out-prompt}Out\hspace{4pt}{[}5{]}:\hspace{4pt}}\\*
                \vspace{-2.55\baselineskip}\begin{InvisibleVerbatim}
                \vspace{-0.5\baselineskip}
\begin{alltt}Int64\end{alltt}

            \end{InvisibleVerbatim}
            
        
    
Because this is not especially readable, I wrote a helper function to
pull out the return type:

    % Make sure that atleast 4 lines are below the HR
    \needspace{4\baselineskip}

    
        \vspace{6pt}
        \makebox[0.1\linewidth]{\smaller\hfill\tt\color{nbframe-in-prompt}In\hspace{4pt}{[}6{]}:\hspace{4pt}}\\*
        \vspace{-2.65\baselineskip}
        \begin{ColorVerbatim}
            \vspace{-0.7\baselineskip}
            \begin{Verbatim}[commandchars=\\\{\}]
\PY{n}{returntype}\PY{p}{(}\PY{n}{e}\PY{p}{:}\PY{p}{:}\PY{n}{Expr}\PY{p}{)} \PY{o}{=}  \PY{n}{e}\PY{o}{.}\PY{n}{args}\PY{p}{[}\PY{l+m+mi}{3}\PY{p}{]}\PY{o}{.}\PY{n}{typ}
\end{Verbatim}

            
                \vspace{-0.2\baselineskip}
            
        \end{ColorVerbatim}
    

    

        % If the first block is an image, minipage the image.  Else
        % request a certain amount of space for the input text.
        \needspace{4\baselineskip}
        
        

            % Add document contents.
            
                \makebox[0.1\linewidth]{\smaller\hfill\tt\color{nbframe-out-prompt}Out\hspace{4pt}{[}6{]}:\hspace{4pt}}\\*
                \vspace{-2.55\baselineskip}\begin{InvisibleVerbatim}
                \vspace{-0.5\baselineskip}
\begin{alltt}returntype (generic function with 1 method)\end{alltt}

            \end{InvisibleVerbatim}
            
        
    
\subsubsection{Usage examples:}For a call to \texttt{code\_typed} that we know will have one method
returned:

    % Make sure that atleast 4 lines are below the HR
    \needspace{4\baselineskip}

    
        \vspace{6pt}
        \makebox[0.1\linewidth]{\smaller\hfill\tt\color{nbframe-in-prompt}In\hspace{4pt}{[}7{]}:\hspace{4pt}}\\*
        \vspace{-2.65\baselineskip}
        \begin{ColorVerbatim}
            \vspace{-0.7\baselineskip}
            \begin{Verbatim}[commandchars=\\\{\}]
\PY{n}{returntype}\PY{p}{(}\PY{n}{code\PYZus{}typed}\PY{p}{(}\PY{n}{foo}\PY{p}{,}\PY{p}{(}\PY{n}{Int}\PY{p}{,}\PY{p}{)}\PY{p}{)}\PY{p}{[}\PY{l+m+mi}{1}\PY{p}{]}\PY{p}{)}
\end{Verbatim}

            
                \vspace{-0.2\baselineskip}
            
        \end{ColorVerbatim}
    

    

        % If the first block is an image, minipage the image.  Else
        % request a certain amount of space for the input text.
        \needspace{4\baselineskip}
        
        

            % Add document contents.
            
                \makebox[0.1\linewidth]{\smaller\hfill\tt\color{nbframe-out-prompt}Out\hspace{4pt}{[}7{]}:\hspace{4pt}}\\*
                \vspace{-2.55\baselineskip}\begin{InvisibleVerbatim}
                \vspace{-0.5\baselineskip}
\begin{alltt}Int64\end{alltt}

            \end{InvisibleVerbatim}
            
        
    
For calls that might have more than one result:

    % Make sure that atleast 4 lines are below the HR
    \needspace{4\baselineskip}

    
        \vspace{6pt}
        \makebox[0.1\linewidth]{\smaller\hfill\tt\color{nbframe-in-prompt}In\hspace{4pt}{[}8{]}:\hspace{4pt}}\\*
        \vspace{-2.65\baselineskip}
        \begin{ColorVerbatim}
            \vspace{-0.7\baselineskip}
            \begin{Verbatim}[commandchars=\\\{\}]
\PY{n}{Type}\PY{p}{[}\PY{n}{returntype}\PY{p}{(}\PY{n}{t}\PY{p}{)} \PY{k}{for} \PY{n}{t} \PY{o+ow}{in} \PY{n}{code\PYZus{}typed}\PY{p}{(}\PY{o}{+}\PY{p}{,}\PY{p}{(}\PY{n}{Number}\PY{p}{,}\PY{p}{)}\PY{p}{)}\PY{p}{]}
\end{Verbatim}

            
                \vspace{-0.2\baselineskip}
            
        \end{ColorVerbatim}
    

    

        % If the first block is an image, minipage the image.  Else
        % request a certain amount of space for the input text.
        \needspace{4\baselineskip}
        
        

            % Add document contents.
            
                \makebox[0.1\linewidth]{\smaller\hfill\tt\color{nbframe-out-prompt}Out\hspace{4pt}{[}8{]}:\hspace{4pt}}\\*
                \vspace{-2.55\baselineskip}\begin{InvisibleVerbatim}
                \vspace{-0.5\baselineskip}
\begin{alltt}2-element Array\{Type\{T<:Top\},1\}:
 Int64
 Number\end{alltt}

            \end{InvisibleVerbatim}
            
        
    


    % Make sure that atleast 4 lines are below the HR
    \needspace{4\baselineskip}

    
        \vspace{6pt}
        \makebox[0.1\linewidth]{\smaller\hfill\tt\color{nbframe-in-prompt}In\hspace{4pt}{[}9{]}:\hspace{4pt}}\\*
        \vspace{-2.65\baselineskip}
        \begin{ColorVerbatim}
            \vspace{-0.7\baselineskip}
            \begin{Verbatim}[commandchars=\\\{\}]
\PY{n+nb}{map}\PY{p}{(}\PY{n}{returntype}\PY{p}{,}\PY{n}{code\PYZus{}typed}\PY{p}{(}\PY{o}{+}\PY{p}{,}\PY{p}{(}\PY{n}{Any}\PY{p}{,}\PY{n}{Any}\PY{p}{,}\PY{n}{Any}\PY{p}{)}\PY{p}{)}\PY{p}{)}
\end{Verbatim}

            
                \vspace{-0.2\baselineskip}
            
        \end{ColorVerbatim}
    

    

        % If the first block is an image, minipage the image.  Else
        % request a certain amount of space for the input text.
        \needspace{4\baselineskip}
        
        

            % Add document contents.
            
                \makebox[0.1\linewidth]{\smaller\hfill\tt\color{nbframe-out-prompt}Out\hspace{4pt}{[}9{]}:\hspace{4pt}}\\*
                \vspace{-2.55\baselineskip}\begin{InvisibleVerbatim}
                \vspace{-0.5\baselineskip}
\begin{alltt}3-element Array\{Any,1\}:
 BigInt
 BigFloat
 Any\end{alltt}

            \end{InvisibleVerbatim}
            
        
    
\subsection{A Function to Retrieve All Expression in the Function Body}The inner \texttt{Expr} with head \texttt{:body} contains the body of
the function: its \texttt{args} is an array of \texttt{Expr}s. This is
another convenience function to make the code more readable.

    % Make sure that atleast 4 lines are below the HR
    \needspace{4\baselineskip}

    
        \vspace{6pt}
        \makebox[0.1\linewidth]{\smaller\hfill\tt\color{nbframe-in-prompt}In\hspace{4pt}{[}10{]}:\hspace{4pt}}\\*
        \vspace{-2.65\baselineskip}
        \begin{ColorVerbatim}
            \vspace{-0.7\baselineskip}
            \begin{Verbatim}[commandchars=\\\{\}]
\PY{n}{body}\PY{p}{(}\PY{n}{e}\PY{p}{:}\PY{p}{:}\PY{n}{Expr}\PY{p}{)} \PY{o}{=} \PY{n}{e}\PY{o}{.}\PY{n}{args}\PY{p}{[}\PY{l+m+mi}{3}\PY{p}{]}\PY{o}{.}\PY{n}{args}
\end{Verbatim}

            
                \vspace{-0.2\baselineskip}
            
        \end{ColorVerbatim}
    

    

        % If the first block is an image, minipage the image.  Else
        % request a certain amount of space for the input text.
        \needspace{4\baselineskip}
        
        

            % Add document contents.
            
                \makebox[0.1\linewidth]{\smaller\hfill\tt\color{nbframe-out-prompt}Out\hspace{4pt}{[}10{]}:\hspace{4pt}}\\*
                \vspace{-2.55\baselineskip}\begin{InvisibleVerbatim}
                \vspace{-0.5\baselineskip}
\begin{alltt}body (generic function with 1 method)\end{alltt}

            \end{InvisibleVerbatim}
            
        
    
It is used analogously to \texttt{returntype} above.

    % Make sure that atleast 4 lines are below the HR
    \needspace{4\baselineskip}

    
        \vspace{6pt}
        \makebox[0.1\linewidth]{\smaller\hfill\tt\color{nbframe-in-prompt}In\hspace{4pt}{[}11{]}:\hspace{4pt}}\\*
        \vspace{-2.65\baselineskip}
        \begin{ColorVerbatim}
            \vspace{-0.7\baselineskip}
            \begin{Verbatim}[commandchars=\\\{\}]
\PY{n}{body}\PY{p}{(}\PY{n}{code\PYZus{}typed}\PY{p}{(}\PY{n}{foo}\PY{p}{,}\PY{p}{(}\PY{n}{Int}\PY{p}{,}\PY{p}{)}\PY{p}{)}\PY{p}{[}\PY{l+m+mi}{1}\PY{p}{]}\PY{p}{)}
\end{Verbatim}

            
                \vspace{-0.2\baselineskip}
            
        \end{ColorVerbatim}
    

    

        % If the first block is an image, minipage the image.  Else
        % request a certain amount of space for the input text.
        \needspace{4\baselineskip}
        
        

            % Add document contents.
            
                \makebox[0.1\linewidth]{\smaller\hfill\tt\color{nbframe-out-prompt}Out\hspace{4pt}{[}11{]}:\hspace{4pt}}\\*
                \vspace{-2.55\baselineskip}\begin{InvisibleVerbatim}
                \vspace{-0.5\baselineskip}
\begin{alltt}4-element Array\{Any,1\}:
 :( \# In[2], line 2:)
 :(z = top(box)(Int64,top(add\_int)(x::Int64,5))::Int64)
 :( \# line 3:)
 :(return top(box)(Int64,top(mul\_int)(2,z::Int64))::Int64)\end{alltt}

            \end{InvisibleVerbatim}
            
        
    
\subsection{A Function to Retrieve All Return Statements From a Function}\texttt{Expr}s that represent return statements have \texttt{head} set
to \texttt{:return}, so the below function pulls them out of the body.

    % Make sure that atleast 4 lines are below the HR
    \needspace{4\baselineskip}

    
        \vspace{6pt}
        \makebox[0.1\linewidth]{\smaller\hfill\tt\color{nbframe-in-prompt}In\hspace{4pt}{[}12{]}:\hspace{4pt}}\\*
        \vspace{-2.65\baselineskip}
        \begin{ColorVerbatim}
            \vspace{-0.7\baselineskip}
            \begin{Verbatim}[commandchars=\\\{\}]
\PY{n}{returns}\PY{p}{(}\PY{n}{e}\PY{p}{:}\PY{p}{:}\PY{n}{Expr}\PY{p}{)} \PY{o}{=} \PY{n+nb}{filter}\PY{p}{(}\PY{n}{x}\PY{o}{\PYZhy{}}\PY{o}{\PYZgt{}} \PY{n}{typeof}\PY{p}{(}\PY{n}{x}\PY{p}{)} \PY{o}{==} \PY{n}{Expr} \PY{o}{\PYZam{}}\PY{o}{\PYZam{}} \PY{n}{x}\PY{o}{.}\PY{n}{head}\PY{o}{==}\PY{p}{:}\PY{k}{return}\PY{p}{,}\PY{n}{body}\PY{p}{(}\PY{n}{e}\PY{p}{)}\PY{p}{)}
\end{Verbatim}

            
                \vspace{-0.2\baselineskip}
            
        \end{ColorVerbatim}
    

    

        % If the first block is an image, minipage the image.  Else
        % request a certain amount of space for the input text.
        \needspace{4\baselineskip}
        
        

            % Add document contents.
            
                \makebox[0.1\linewidth]{\smaller\hfill\tt\color{nbframe-out-prompt}Out\hspace{4pt}{[}12{]}:\hspace{4pt}}\\*
                \vspace{-2.55\baselineskip}\begin{InvisibleVerbatim}
                \vspace{-0.5\baselineskip}
\begin{alltt}returns (generic function with 1 method)\end{alltt}

            \end{InvisibleVerbatim}
            
        
    


    % Make sure that atleast 4 lines are below the HR
    \needspace{4\baselineskip}

    
        \vspace{6pt}
        \makebox[0.1\linewidth]{\smaller\hfill\tt\color{nbframe-in-prompt}In\hspace{4pt}{[}13{]}:\hspace{4pt}}\\*
        \vspace{-2.65\baselineskip}
        \begin{ColorVerbatim}
            \vspace{-0.7\baselineskip}
            \begin{Verbatim}[commandchars=\\\{\}]
\PY{n}{returns}\PY{p}{(}\PY{n}{code\PYZus{}typed}\PY{p}{(}\PY{n}{foo}\PY{p}{,}\PY{p}{(}\PY{n}{Int}\PY{p}{,}\PY{p}{)}\PY{p}{)}\PY{p}{[}\PY{l+m+mi}{1}\PY{p}{]}\PY{p}{)}
\end{Verbatim}

            
                \vspace{-0.2\baselineskip}
            
        \end{ColorVerbatim}
    

    

        % If the first block is an image, minipage the image.  Else
        % request a certain amount of space for the input text.
        \needspace{4\baselineskip}
        
        

            % Add document contents.
            
                \makebox[0.1\linewidth]{\smaller\hfill\tt\color{nbframe-out-prompt}Out\hspace{4pt}{[}13{]}:\hspace{4pt}}\\*
                \vspace{-2.55\baselineskip}\begin{InvisibleVerbatim}
                \vspace{-0.5\baselineskip}
\begin{alltt}1-element Array\{Any,1\}:
 :(return top(box)(Int64,top(mul\_int)(2,z::Int64))::Int64)\end{alltt}

            \end{InvisibleVerbatim}
            
        
    


    % Make sure that atleast 4 lines are below the HR
    \needspace{4\baselineskip}

    
        \vspace{6pt}
        \makebox[0.1\linewidth]{\smaller\hfill\tt\color{nbframe-in-prompt}In\hspace{4pt}{[}14{]}:\hspace{4pt}}\\*
        \vspace{-2.65\baselineskip}
        \begin{ColorVerbatim}
            \vspace{-0.7\baselineskip}
            \begin{Verbatim}[commandchars=\\\{\}]
\PY{n}{function} \PY{n}{barr}\PY{p}{(}\PY{n}{x}\PY{p}{:}\PY{p}{:}\PY{n}{Int}\PY{p}{)}
    \PY{n}{x} \PY{o}{+} \PY{l+m+mi}{2}
\PY{n}{end}

\PY{n}{returns}\PY{p}{(}\PY{n}{code\PYZus{}typed}\PY{p}{(}\PY{n}{barr}\PY{p}{,}\PY{p}{(}\PY{n}{Int}\PY{p}{,}\PY{p}{)}\PY{p}{)}\PY{p}{[}\PY{l+m+mi}{1}\PY{p}{]}\PY{p}{)}
\end{Verbatim}

            
                \vspace{-0.2\baselineskip}
            
        \end{ColorVerbatim}
    

    

        % If the first block is an image, minipage the image.  Else
        % request a certain amount of space for the input text.
        \needspace{4\baselineskip}
        
        

            % Add document contents.
            
                \makebox[0.1\linewidth]{\smaller\hfill\tt\color{nbframe-out-prompt}Out\hspace{4pt}{[}14{]}:\hspace{4pt}}\\*
                \vspace{-2.55\baselineskip}\begin{InvisibleVerbatim}
                \vspace{-0.5\baselineskip}
\begin{alltt}1-element Array\{Any,1\}:
 :(return top(box)(Int64,top(add\_int)(x::Int64,2))::Int64)\end{alltt}

            \end{InvisibleVerbatim}
            
        
    
Notice that we still get a \texttt{:return}, even if we don't use the
keyword \texttt{return}. The last expression in a function becomes the
return value if there is no \texttt{return}, and this is expressed in
the AST by desugaring to a normal \texttt{:return}.\subsection{Other Helper Functions}My project resulted in the \texttt{TypeCheck.jl} package for Julia. It
is unlikely to be worth the space to explain the implementation of all
the other helper functions I wrote here. I'll use functions from
\texttt{TypeCheck} later, with comments explaining briefly what they do.\part{Checking for Stable Return Types}It is good style in Julia for the return type of a method to only depend
on the types of the arguments and not on their values. This stability
makes behavior more predictable for programmers. It also allows type
inference to work better -- stable types on called methods allows stable
types on the variables you put the return values into.The following method is a simple example of an unstable return type.
Sometimes it returns an \texttt{Int} and sometimes a \texttt{Bool}. The
return type of this method would be inferred as
\texttt{Union(Int64,Bool)}.

    % Make sure that atleast 4 lines are below the HR
    \needspace{4\baselineskip}

    
        \vspace{6pt}
        \makebox[0.1\linewidth]{\smaller\hfill\tt\color{nbframe-in-prompt}In\hspace{4pt}{[}15{]}:\hspace{4pt}}\\*
        \vspace{-2.65\baselineskip}
        \begin{ColorVerbatim}
            \vspace{-0.7\baselineskip}
            \begin{Verbatim}[commandchars=\\\{\}]
\PY{n}{function} \PY{n}{unstable}\PY{p}{(}\PY{n}{x}\PY{p}{:}\PY{p}{:}\PY{n}{Int}\PY{p}{)}
  \PY{k}{if} \PY{n}{x} \PY{o}{\PYZgt{}} \PY{l+m+mi}{5}
    \PY{k}{return} \PY{n}{x}
  \PY{k}{else}
    \PY{k}{return} \PY{n}{false}
  \PY{n}{end}
\PY{n}{end}
\end{Verbatim}

            
                \vspace{-0.2\baselineskip}
            
        \end{ColorVerbatim}
    

    

        % If the first block is an image, minipage the image.  Else
        % request a certain amount of space for the input text.
        \needspace{4\baselineskip}
        
        

            % Add document contents.
            
                \makebox[0.1\linewidth]{\smaller\hfill\tt\color{nbframe-out-prompt}Out\hspace{4pt}{[}15{]}:\hspace{4pt}}\\*
                \vspace{-2.55\baselineskip}\begin{InvisibleVerbatim}
                \vspace{-0.5\baselineskip}
\begin{alltt}unstable (generic function with 1 method)\end{alltt}

            \end{InvisibleVerbatim}
            
        
    


    % Make sure that atleast 4 lines are below the HR
    \needspace{4\baselineskip}

    
        \vspace{6pt}
        \makebox[0.1\linewidth]{\smaller\hfill\tt\color{nbframe-in-prompt}In\hspace{4pt}{[}16{]}:\hspace{4pt}}\\*
        \vspace{-2.65\baselineskip}
        \begin{ColorVerbatim}
            \vspace{-0.7\baselineskip}
            \begin{Verbatim}[commandchars=\\\{\}]
\PY{n}{unstable}\PY{p}{(}\PY{l+m+mi}{5}\PY{p}{)}
\end{Verbatim}

            
                \vspace{-0.2\baselineskip}
            
        \end{ColorVerbatim}
    

    

        % If the first block is an image, minipage the image.  Else
        % request a certain amount of space for the input text.
        \needspace{4\baselineskip}
        
        

            % Add document contents.
            
                \makebox[0.1\linewidth]{\smaller\hfill\tt\color{nbframe-out-prompt}Out\hspace{4pt}{[}16{]}:\hspace{4pt}}\\*
                \vspace{-2.55\baselineskip}\begin{InvisibleVerbatim}
                \vspace{-0.5\baselineskip}
\begin{alltt}false\end{alltt}

            \end{InvisibleVerbatim}
            
        
    


    % Make sure that atleast 4 lines are below the HR
    \needspace{4\baselineskip}

    
        \vspace{6pt}
        \makebox[0.1\linewidth]{\smaller\hfill\tt\color{nbframe-in-prompt}In\hspace{4pt}{[}17{]}:\hspace{4pt}}\\*
        \vspace{-2.65\baselineskip}
        \begin{ColorVerbatim}
            \vspace{-0.7\baselineskip}
            \begin{Verbatim}[commandchars=\\\{\}]
\PY{n}{unstable}\PY{p}{(}\PY{l+m+mi}{1337}\PY{p}{)}
\end{Verbatim}

            
                \vspace{-0.2\baselineskip}
            
        \end{ColorVerbatim}
    

    

        % If the first block is an image, minipage the image.  Else
        % request a certain amount of space for the input text.
        \needspace{4\baselineskip}
        
        

            % Add document contents.
            
                \makebox[0.1\linewidth]{\smaller\hfill\tt\color{nbframe-out-prompt}Out\hspace{4pt}{[}17{]}:\hspace{4pt}}\\*
                \vspace{-2.55\baselineskip}\begin{InvisibleVerbatim}
                \vspace{-0.5\baselineskip}
\begin{alltt}1337\end{alltt}

            \end{InvisibleVerbatim}
            
        
    
Until now, there has been no way to automatically check that methods do
not behave in this way. Julia's base library is mostly free of this,
through the use of code review. While there are instances of
instability, they tend to be less obvious -- they stem especially from
retrieving data from untyped storage, from some interfaces to other
environments, or from places where it is necessary (higher-level
functions).I have written a static checker to detect that this invariant may be
violated. My approach tends more towards false positives than false
negatives.\section{Deciding Whether the Return Type is Probably Stable}If the types of an argument to a method are all concrete, then the
return type should also be concrete.\subsection{Concrete, Abstract, and Union Types}I've already mentioned concrete and abstract types, which are the leaves
and internal nodes of the type hierarchy, respectively. Union types are
collection of types. They are similar to abstract types in that they
have subtypes, but they do not have names and do not alter the type
hierarchy. Union types provide a way to say ``Any of these types or
their subtypes''.

    % Make sure that atleast 4 lines are below the HR
    \needspace{4\baselineskip}

    
        \vspace{6pt}
        \makebox[0.1\linewidth]{\smaller\hfill\tt\color{nbframe-in-prompt}In\hspace{4pt}{[}18{]}:\hspace{4pt}}\\*
        \vspace{-2.65\baselineskip}
        \begin{ColorVerbatim}
            \vspace{-0.7\baselineskip}
            \begin{Verbatim}[commandchars=\\\{\}]
\PY{p}{[}\PY{n}{Int}\PY{p}{,} \PY{n}{String}\PY{p}{,} \PY{n}{Union}\PY{p}{(}\PY{n}{Float64}\PY{p}{,}\PY{n}{UTF8String}\PY{p}{)}\PY{p}{]}
\end{Verbatim}

            
                \vspace{-0.2\baselineskip}
            
        \end{ColorVerbatim}
    

    

        % If the first block is an image, minipage the image.  Else
        % request a certain amount of space for the input text.
        \needspace{4\baselineskip}
        
        

            % Add document contents.
            
                \makebox[0.1\linewidth]{\smaller\hfill\tt\color{nbframe-out-prompt}Out\hspace{4pt}{[}18{]}:\hspace{4pt}}\\*
                \vspace{-2.55\baselineskip}\begin{InvisibleVerbatim}
                \vspace{-0.5\baselineskip}
\begin{alltt}3-element Array\{Type\{T<:Top\},1\}:
 Int64
 String
 Union(UTF8String,Float64)\end{alltt}

            \end{InvisibleVerbatim}
            
        
    
Unlike concrete and abstract types, union types are not represented by
\texttt{DataType}; they have their own type, \texttt{UnionType}.

    % Make sure that atleast 4 lines are below the HR
    \needspace{4\baselineskip}

    
        \vspace{6pt}
        \makebox[0.1\linewidth]{\smaller\hfill\tt\color{nbframe-in-prompt}In\hspace{4pt}{[}19{]}:\hspace{4pt}}\\*
        \vspace{-2.65\baselineskip}
        \begin{ColorVerbatim}
            \vspace{-0.7\baselineskip}
            \begin{Verbatim}[commandchars=\\\{\}]
\PY{p}{[}\PY{n}{typeof}\PY{p}{(}\PY{n}{x}\PY{p}{)} \PY{k}{for} \PY{n}{x} \PY{o+ow}{in} \PY{n}{ans}\PY{p}{]}
\end{Verbatim}

            
                \vspace{-0.2\baselineskip}
            
        \end{ColorVerbatim}
    

    

        % If the first block is an image, minipage the image.  Else
        % request a certain amount of space for the input text.
        \needspace{4\baselineskip}
        
        

            % Add document contents.
            
                \makebox[0.1\linewidth]{\smaller\hfill\tt\color{nbframe-out-prompt}Out\hspace{4pt}{[}19{]}:\hspace{4pt}}\\*
                \vspace{-2.55\baselineskip}\begin{InvisibleVerbatim}
                \vspace{-0.5\baselineskip}
\begin{alltt}3-element Array\{Type\{\_\},1\}:
 DataType
 DataType
 UnionType\end{alltt}

            \end{InvisibleVerbatim}
            
        
    
There is a convenient function for differetiating between concrete types
and all other types: \texttt{isleaftype}. It returns true for concrete
types (the leaves of the type hierarchy) and false for abstract types
and union types.

    % Make sure that atleast 4 lines are below the HR
    \needspace{4\baselineskip}

    
        \vspace{6pt}
        \makebox[0.1\linewidth]{\smaller\hfill\tt\color{nbframe-in-prompt}In\hspace{4pt}{[}20{]}:\hspace{4pt}}\\*
        \vspace{-2.65\baselineskip}
        \begin{ColorVerbatim}
            \vspace{-0.7\baselineskip}
            \begin{Verbatim}[commandchars=\\\{\}]
\PY{n}{isleaftype}\PY{p}{(}\PY{n}{Uint128}\PY{p}{)}
\end{Verbatim}

            
                \vspace{-0.2\baselineskip}
            
        \end{ColorVerbatim}
    

    

        % If the first block is an image, minipage the image.  Else
        % request a certain amount of space for the input text.
        \needspace{4\baselineskip}
        
        

            % Add document contents.
            
                \makebox[0.1\linewidth]{\smaller\hfill\tt\color{nbframe-out-prompt}Out\hspace{4pt}{[}20{]}:\hspace{4pt}}\\*
                \vspace{-2.55\baselineskip}\begin{InvisibleVerbatim}
                \vspace{-0.5\baselineskip}
\begin{alltt}true\end{alltt}

            \end{InvisibleVerbatim}
            
        
    


    % Make sure that atleast 4 lines are below the HR
    \needspace{4\baselineskip}

    
        \vspace{6pt}
        \makebox[0.1\linewidth]{\smaller\hfill\tt\color{nbframe-in-prompt}In\hspace{4pt}{[}21{]}:\hspace{4pt}}\\*
        \vspace{-2.65\baselineskip}
        \begin{ColorVerbatim}
            \vspace{-0.7\baselineskip}
            \begin{Verbatim}[commandchars=\\\{\}]
\PY{n}{isleaftype}\PY{p}{(}\PY{n}{String}\PY{p}{)}
\end{Verbatim}

            
                \vspace{-0.2\baselineskip}
            
        \end{ColorVerbatim}
    

    

        % If the first block is an image, minipage the image.  Else
        % request a certain amount of space for the input text.
        \needspace{4\baselineskip}
        
        

            % Add document contents.
            
                \makebox[0.1\linewidth]{\smaller\hfill\tt\color{nbframe-out-prompt}Out\hspace{4pt}{[}21{]}:\hspace{4pt}}\\*
                \vspace{-2.55\baselineskip}\begin{InvisibleVerbatim}
                \vspace{-0.5\baselineskip}
\begin{alltt}false\end{alltt}

            \end{InvisibleVerbatim}
            
        
    


    % Make sure that atleast 4 lines are below the HR
    \needspace{4\baselineskip}

    
        \vspace{6pt}
        \makebox[0.1\linewidth]{\smaller\hfill\tt\color{nbframe-in-prompt}In\hspace{4pt}{[}22{]}:\hspace{4pt}}\\*
        \vspace{-2.65\baselineskip}
        \begin{ColorVerbatim}
            \vspace{-0.7\baselineskip}
            \begin{Verbatim}[commandchars=\\\{\}]
\PY{n}{isleaftype}\PY{p}{(}\PY{n}{Union}\PY{p}{(}\PY{n}{Int}\PY{p}{,}\PY{n}{Float32}\PY{p}{,}\PY{n}{ASCIIString}\PY{p}{)}\PY{p}{)}
\end{Verbatim}

            
                \vspace{-0.2\baselineskip}
            
        \end{ColorVerbatim}
    

    

        % If the first block is an image, minipage the image.  Else
        % request a certain amount of space for the input text.
        \needspace{4\baselineskip}
        
        

            % Add document contents.
            
                \makebox[0.1\linewidth]{\smaller\hfill\tt\color{nbframe-out-prompt}Out\hspace{4pt}{[}22{]}:\hspace{4pt}}\\*
                \vspace{-2.55\baselineskip}\begin{InvisibleVerbatim}
                \vspace{-0.5\baselineskip}
\begin{alltt}false\end{alltt}

            \end{InvisibleVerbatim}
            
        
    
\subsection{The Basic Check}Given an \texttt{Expr} from \texttt{code\_typed}, we want to grab the
types of the arguments and the return type. If the return type is
concrete, then everything is fine: a concrete type can't be unstable. If
the return type is not concrete and at least one argument is not
concrete, then I don't warn about that method: the cause could be the
possible types of the non-concrete argument types.

    % Make sure that atleast 4 lines are below the HR
    \needspace{4\baselineskip}

    
        \vspace{6pt}
        \makebox[0.1\linewidth]{\smaller\hfill\tt\color{nbframe-in-prompt}In\hspace{4pt}{[}23{]}:\hspace{4pt}}\\*
        \vspace{-2.65\baselineskip}
        \begin{ColorVerbatim}
            \vspace{-0.7\baselineskip}
            \begin{Verbatim}[commandchars=\\\{\}]
\PY{n}{using} \PY{n}{TypeCheck}
\PY{n}{function} \PY{n}{isreturnbasedonvalues}\PY{p}{(}\PY{n}{e}\PY{p}{:}\PY{p}{:}\PY{n}{Expr}\PY{p}{)}
  \PY{n}{rt} \PY{o}{=} \PY{n}{returntype}\PY{p}{(}\PY{n}{e}\PY{p}{)}
  \PY{n}{ts} \PY{o}{=} \PY{n}{TypeCheck}\PY{o}{.}\PY{n}{argtypes}\PY{p}{(}\PY{n}{e}\PY{p}{)} \PY{c}{\PYZsh{}the type of each argument in e\PYZsq{}s type signature}

  \PY{k}{if} \PY{n}{isleaftype}\PY{p}{(}\PY{n}{rt}\PY{p}{)} \PY{o}{|}\PY{o}{|} \PY{n}{rt} \PY{o}{==} \PY{n+nb+bp}{None}
    \PY{k}{return} \PY{n}{false}
  \PY{n}{end}

  \PY{k}{for} \PY{n}{t} \PY{o+ow}{in} \PY{n}{ts}
   \PY{k}{if} \PY{err}{!}\PY{n}{isleaftype}\PY{p}{(}\PY{n}{t}\PY{p}{)}
     \PY{k}{return} \PY{n}{false}
   \PY{n}{end}
  \PY{n}{end}

  \PY{k}{return} \PY{n}{true} \PY{c}{\PYZsh{} return is not concrete type; all args are concrete types}
\PY{n}{end}
\end{Verbatim}

            
                \vspace{-0.2\baselineskip}
            
        \end{ColorVerbatim}
    

    

        % If the first block is an image, minipage the image.  Else
        % request a certain amount of space for the input text.
        \needspace{4\baselineskip}
        
        

            % Add document contents.
            
                \makebox[0.1\linewidth]{\smaller\hfill\tt\color{nbframe-out-prompt}Out\hspace{4pt}{[}23{]}:\hspace{4pt}}\\*
                \vspace{-2.55\baselineskip}\begin{InvisibleVerbatim}
                \vspace{-0.5\baselineskip}
\begin{alltt}isreturnbasedonvalues (generic function with 1 method)\end{alltt}

            \end{InvisibleVerbatim}
            
        
    


    % Make sure that atleast 4 lines are below the HR
    \needspace{4\baselineskip}

    
        \vspace{6pt}
        \makebox[0.1\linewidth]{\smaller\hfill\tt\color{nbframe-in-prompt}In\hspace{4pt}{[}24{]}:\hspace{4pt}}\\*
        \vspace{-2.65\baselineskip}
        \begin{ColorVerbatim}
            \vspace{-0.7\baselineskip}
            \begin{Verbatim}[commandchars=\\\{\}]
\PY{n}{isreturnbasedonvalues}\PY{p}{(}\PY{n}{code\PYZus{}typed}\PY{p}{(}\PY{n}{unstable}\PY{p}{,}\PY{p}{(}\PY{n}{Int}\PY{p}{,}\PY{p}{)}\PY{p}{)}\PY{p}{[}\PY{l+m+mi}{1}\PY{p}{]}\PY{p}{)}
\end{Verbatim}

            
                \vspace{-0.2\baselineskip}
            
        \end{ColorVerbatim}
    

    

        % If the first block is an image, minipage the image.  Else
        % request a certain amount of space for the input text.
        \needspace{4\baselineskip}
        
        

            % Add document contents.
            
                \makebox[0.1\linewidth]{\smaller\hfill\tt\color{nbframe-out-prompt}Out\hspace{4pt}{[}24{]}:\hspace{4pt}}\\*
                \vspace{-2.55\baselineskip}\begin{InvisibleVerbatim}
                \vspace{-0.5\baselineskip}
\begin{alltt}true\end{alltt}

            \end{InvisibleVerbatim}
            
        
    


    % Make sure that atleast 4 lines are below the HR
    \needspace{4\baselineskip}

    
        \vspace{6pt}
        \makebox[0.1\linewidth]{\smaller\hfill\tt\color{nbframe-in-prompt}In\hspace{4pt}{[}25{]}:\hspace{4pt}}\\*
        \vspace{-2.65\baselineskip}
        \begin{ColorVerbatim}
            \vspace{-0.7\baselineskip}
            \begin{Verbatim}[commandchars=\\\{\}]
\PY{n}{isreturnbasedonvalues}\PY{p}{(}\PY{n}{code\PYZus{}typed}\PY{p}{(}\PY{n}{foo}\PY{p}{,}\PY{p}{(}\PY{n}{Int}\PY{p}{,}\PY{p}{)}\PY{p}{)}\PY{p}{[}\PY{l+m+mi}{1}\PY{p}{]}\PY{p}{)}
\end{Verbatim}

            
                \vspace{-0.2\baselineskip}
            
        \end{ColorVerbatim}
    

    

        % If the first block is an image, minipage the image.  Else
        % request a certain amount of space for the input text.
        \needspace{4\baselineskip}
        
        

            % Add document contents.
            
                \makebox[0.1\linewidth]{\smaller\hfill\tt\color{nbframe-out-prompt}Out\hspace{4pt}{[}25{]}:\hspace{4pt}}\\*
                \vspace{-2.55\baselineskip}\begin{InvisibleVerbatim}
                \vspace{-0.5\baselineskip}
\begin{alltt}false\end{alltt}

            \end{InvisibleVerbatim}
            
        
    
While this check works on simple examples, it tends to have many
false-positives when running on large modules, such as the standard
library.\section{Preventing One Unstable Function from Spawning Many More Warnings}With the simple function above, I get a lot of warnings on the base
library. When I dug into the causes of some of them, there was a
frequent pattern. Many functions would trigger warnings because their
return type depended on a call to another function. A small handful of
functions would actually need changes to become type-stable, but they
are lost in the sea of their users.

These users are sort of ``semi-false-positives'': their return type is
actually unstable, but it's not their fault -- and the change probably
shouldn't happen there. Most of these functions can be filtered out by
looking at the \texttt{:return}s in the function body and letting them
pass if their return type is determined by \texttt{:call}s to other
functions (which are unstable).

To be more specific, if \texttt{foo} calls \texttt{barr}, and
\texttt{barr} is unstable, then I would like to only warn the user about
\texttt{barr}, not about \texttt{foo}. This simple check will only work
if \texttt{foo}'s final expression or \texttt{return} expression is a
call to \texttt{barr}. Somewhat surprisingly, this seems to work for
basically all of the standard library.\subsection{A Example of Return Type Propogation}Here, \texttt{f1} is unstable. We should make a change to \texttt{f1} if
we want it's return type to be stable.
\texttt{isreturntypebasedonvalues} would also warn about \texttt{f2}.
However, \texttt{f2} just calls \texttt{f1}, so there's not necessarily
a change to be made to \texttt{f2}.

    % Make sure that atleast 4 lines are below the HR
    \needspace{4\baselineskip}

    
        \vspace{6pt}
        \makebox[0.1\linewidth]{\smaller\hfill\tt\color{nbframe-in-prompt}In\hspace{4pt}{[}31{]}:\hspace{4pt}}\\*
        \vspace{-2.65\baselineskip}
        \begin{ColorVerbatim}
            \vspace{-0.7\baselineskip}
            \begin{Verbatim}[commandchars=\\\{\}]
\PY{n}{f1}\PY{p}{(}\PY{n}{x}\PY{p}{:}\PY{p}{:}\PY{n}{Int}\PY{p}{)} \PY{o}{=} \PY{n}{x} \PY{o}{==} \PY{l+m+mi}{5} \PY{err}{?} \PY{l+m+mi}{42} \PY{p}{:} \PY{n}{pi}
\PY{n}{returntype}\PY{p}{(}\PY{n}{code\PYZus{}typed}\PY{p}{(}\PY{n}{f1}\PY{p}{,}\PY{p}{(}\PY{n}{Int}\PY{p}{,}\PY{p}{)}\PY{p}{)}\PY{p}{[}\PY{l+m+mi}{1}\PY{p}{]}\PY{p}{)}
\end{Verbatim}

            
                \vspace{-0.2\baselineskip}
            
        \end{ColorVerbatim}
    

    

        % If the first block is an image, minipage the image.  Else
        % request a certain amount of space for the input text.
        \needspace{4\baselineskip}
        
        

            % Add document contents.
            
                \makebox[0.1\linewidth]{\smaller\hfill\tt\color{nbframe-out-prompt}Out\hspace{4pt}{[}31{]}:\hspace{4pt}}\\*
                \vspace{-2.55\baselineskip}\begin{InvisibleVerbatim}
                \vspace{-0.5\baselineskip}
\begin{alltt}Union(Int64,MathConst\{:π\})\end{alltt}

            \end{InvisibleVerbatim}
            
        
    


    % Make sure that atleast 4 lines are below the HR
    \needspace{4\baselineskip}

    
        \vspace{6pt}
        \makebox[0.1\linewidth]{\smaller\hfill\tt\color{nbframe-in-prompt}In\hspace{4pt}{[}32{]}:\hspace{4pt}}\\*
        \vspace{-2.65\baselineskip}
        \begin{ColorVerbatim}
            \vspace{-0.7\baselineskip}
            \begin{Verbatim}[commandchars=\\\{\}]
\PY{n}{f2}\PY{p}{(}\PY{n}{y}\PY{p}{:}\PY{p}{:}\PY{n}{Int}\PY{p}{)} \PY{o}{=} \PY{n}{f1}\PY{p}{(}\PY{n}{y} \PY{o}{+} \PY{l+m+mi}{2}\PY{p}{)}
\PY{n}{returntype}\PY{p}{(}\PY{n}{code\PYZus{}typed}\PY{p}{(}\PY{n}{f2}\PY{p}{,}\PY{p}{(}\PY{n}{Int}\PY{p}{,}\PY{p}{)}\PY{p}{)}\PY{p}{[}\PY{l+m+mi}{1}\PY{p}{]}\PY{p}{)}
\end{Verbatim}

            
                \vspace{-0.2\baselineskip}
            
        \end{ColorVerbatim}
    

    

        % If the first block is an image, minipage the image.  Else
        % request a certain amount of space for the input text.
        \needspace{4\baselineskip}
        
        

            % Add document contents.
            
                \makebox[0.1\linewidth]{\smaller\hfill\tt\color{nbframe-out-prompt}Out\hspace{4pt}{[}32{]}:\hspace{4pt}}\\*
                \vspace{-2.55\baselineskip}\begin{InvisibleVerbatim}
                \vspace{-0.5\baselineskip}
\begin{alltt}Union(Int64,MathConst\{:π\})\end{alltt}

            \end{InvisibleVerbatim}
            
        
    
\subsection{Preventing Propogation}We can add this to \texttt{isreturntypebasedonvalues} by giving failing
functions a second chance. This means we can add another check after the
loop that looks for non-concrete argument types.

    % Make sure that atleast 4 lines are below the HR
    \needspace{4\baselineskip}

    
        \vspace{6pt}
        \makebox[0.1\linewidth]{\smaller\hfill\tt\color{nbframe-in-prompt}In\hspace{4pt}{[}41{]}:\hspace{4pt}}\\*
        \vspace{-2.65\baselineskip}
        \begin{ColorVerbatim}
            \vspace{-0.7\baselineskip}
            \begin{Verbatim}[commandchars=\\\{\}]
\PY{n}{function} \PY{n}{isreturnbasedonvalues}\PY{p}{(}\PY{n}{e}\PY{p}{:}\PY{p}{:}\PY{n}{Expr}\PY{p}{)}
    \PY{n}{rt} \PY{o}{=} \PY{n}{returntype}\PY{p}{(}\PY{n}{e}\PY{p}{)}
    \PY{n}{ts} \PY{o}{=} \PY{n}{TypeCheck}\PY{o}{.}\PY{n}{argtypes}\PY{p}{(}\PY{n}{e}\PY{p}{)}

    \PY{k}{if} \PY{n}{isleaftype}\PY{p}{(}\PY{n}{rt}\PY{p}{)} \PY{o}{|}\PY{o}{|} \PY{n}{rt} \PY{o}{==} \PY{n+nb+bp}{None}
      \PY{k}{return} \PY{n}{false}
    \PY{n}{end}
    \PY{k}{for} \PY{n}{t} \PY{o+ow}{in} \PY{n}{ts}
     \PY{k}{if} \PY{err}{!}\PY{n}{isleaftype}\PY{p}{(}\PY{n}{t}\PY{p}{)}
       \PY{k}{return} \PY{n}{false}
     \PY{n}{end}
    \PY{n}{end}

    \PY{c}{\PYZsh{}a second chance}
    \PY{c}{\PYZsh{}cs is a list of return types for calls in :return exprs in e\PYZsq{}s body}
    \PY{n}{cs} \PY{o}{=} \PY{p}{[}\PY{n}{TypeCheck}\PY{o}{.}\PY{n}{find\PYZus{}returntype}\PY{p}{(}\PY{n}{c}\PY{p}{,}\PY{n}{e}\PY{p}{)} \PY{k}{for} \PY{n}{c} \PY{o+ow}{in} \PY{n}{TypeCheck}\PY{o}{.}\PY{n}{extract\PYZus{}calls\PYZus{}from\PYZus{}returns}\PY{p}{(}\PY{n}{e}\PY{p}{)}\PY{p}{]}
    \PY{k}{for} \PY{n}{c} \PY{o+ow}{in} \PY{n}{cs}
     \PY{k}{if} \PY{n}{rt} \PY{o}{==} \PY{n}{c} \PY{c}{\PYZsh{}if e\PYZsq{}s return type is the same as a call it\PYZsq{}s making, then it passes}
         \PY{k}{return} \PY{n}{false}
      \PY{n}{end}
    \PY{n}{end}

    \PY{k}{return} \PY{n}{true} \PY{c}{\PYZsh{}e fails the test}
  \PY{n}{end}
\end{Verbatim}

            
                \vspace{-0.2\baselineskip}
            
        \end{ColorVerbatim}
    

    

        % If the first block is an image, minipage the image.  Else
        % request a certain amount of space for the input text.
        \needspace{4\baselineskip}
        
        

            % Add document contents.
            
                \makebox[0.1\linewidth]{\smaller\hfill\tt\color{nbframe-out-prompt}Out\hspace{4pt}{[}41{]}:\hspace{4pt}}\\*
                \vspace{-2.55\baselineskip}\begin{InvisibleVerbatim}
                \vspace{-0.5\baselineskip}
\begin{alltt}isreturnbasedonvalues (generic function with 1 method)\end{alltt}

            \end{InvisibleVerbatim}
            
        
    
Above, I use two new helper functions \texttt{find\_returntype} and
\texttt{extract\_calls\_from\_returns}.
\texttt{extract\_calls\_from\_returns} is not expecially interesting; it
just examines the insides of \texttt{:return} \texttt{Expr}s for
\texttt{:call} \texttt{Expr}s. \texttt{find\_returntype} is more
interesting, and I will explain it below.\subsubsection{Results:}

    % Make sure that atleast 4 lines are below the HR
    \needspace{4\baselineskip}

    
        \vspace{6pt}
        \makebox[0.1\linewidth]{\smaller\hfill\tt\color{nbframe-in-prompt}In\hspace{4pt}{[}37{]}:\hspace{4pt}}\\*
        \vspace{-2.65\baselineskip}
        \begin{ColorVerbatim}
            \vspace{-0.7\baselineskip}
            \begin{Verbatim}[commandchars=\\\{\}]
\PY{n}{isreturnbasedonvalues}\PY{p}{(}\PY{n}{code\PYZus{}typed}\PY{p}{(}\PY{n}{foo}\PY{p}{,}\PY{p}{(}\PY{n}{Int}\PY{p}{,}\PY{p}{)}\PY{p}{)}\PY{p}{[}\PY{l+m+mi}{1}\PY{p}{]}\PY{p}{)} \PY{c}{\PYZsh{}passes}
\end{Verbatim}

            
                \vspace{-0.2\baselineskip}
            
        \end{ColorVerbatim}
    

    

        % If the first block is an image, minipage the image.  Else
        % request a certain amount of space for the input text.
        \needspace{4\baselineskip}
        
        

            % Add document contents.
            
                \makebox[0.1\linewidth]{\smaller\hfill\tt\color{nbframe-out-prompt}Out\hspace{4pt}{[}37{]}:\hspace{4pt}}\\*
                \vspace{-2.55\baselineskip}\begin{InvisibleVerbatim}
                \vspace{-0.5\baselineskip}
\begin{alltt}false\end{alltt}

            \end{InvisibleVerbatim}
            
        
    


    % Make sure that atleast 4 lines are below the HR
    \needspace{4\baselineskip}

    
        \vspace{6pt}
        \makebox[0.1\linewidth]{\smaller\hfill\tt\color{nbframe-in-prompt}In\hspace{4pt}{[}38{]}:\hspace{4pt}}\\*
        \vspace{-2.65\baselineskip}
        \begin{ColorVerbatim}
            \vspace{-0.7\baselineskip}
            \begin{Verbatim}[commandchars=\\\{\}]
\PY{n}{isreturnbasedonvalues}\PY{p}{(}\PY{n}{code\PYZus{}typed}\PY{p}{(}\PY{n}{f1}\PY{p}{,}\PY{p}{(}\PY{n}{Int}\PY{p}{,}\PY{p}{)}\PY{p}{)}\PY{p}{[}\PY{l+m+mi}{1}\PY{p}{]}\PY{p}{)} \PY{c}{\PYZsh{}fails}
\end{Verbatim}

            
                \vspace{-0.2\baselineskip}
            
        \end{ColorVerbatim}
    

    

        % If the first block is an image, minipage the image.  Else
        % request a certain amount of space for the input text.
        \needspace{4\baselineskip}
        
        

            % Add document contents.
            
                \makebox[0.1\linewidth]{\smaller\hfill\tt\color{nbframe-out-prompt}Out\hspace{4pt}{[}38{]}:\hspace{4pt}}\\*
                \vspace{-2.55\baselineskip}\begin{InvisibleVerbatim}
                \vspace{-0.5\baselineskip}
\begin{alltt}true\end{alltt}

            \end{InvisibleVerbatim}
            
        
    


    % Make sure that atleast 4 lines are below the HR
    \needspace{4\baselineskip}

    
        \vspace{6pt}
        \makebox[0.1\linewidth]{\smaller\hfill\tt\color{nbframe-in-prompt}In\hspace{4pt}{[}40{]}:\hspace{4pt}}\\*
        \vspace{-2.65\baselineskip}
        \begin{ColorVerbatim}
            \vspace{-0.7\baselineskip}
            \begin{Verbatim}[commandchars=\\\{\}]
\PY{n}{isreturnbasedonvalues}\PY{p}{(}\PY{n}{code\PYZus{}typed}\PY{p}{(}\PY{n}{f2}\PY{p}{,}\PY{p}{(}\PY{n}{Int}\PY{p}{,}\PY{p}{)}\PY{p}{)}\PY{p}{[}\PY{l+m+mi}{1}\PY{p}{]}\PY{p}{)} \PY{c}{\PYZsh{}passes}
\end{Verbatim}

            
                \vspace{-0.2\baselineskip}
            
        \end{ColorVerbatim}
    

    

        % If the first block is an image, minipage the image.  Else
        % request a certain amount of space for the input text.
        \needspace{4\baselineskip}
        
        

            % Add document contents.
            
                \makebox[0.1\linewidth]{\smaller\hfill\tt\color{nbframe-out-prompt}Out\hspace{4pt}{[}40{]}:\hspace{4pt}}\\*
                \vspace{-2.55\baselineskip}\begin{InvisibleVerbatim}
                \vspace{-0.5\baselineskip}
\begin{alltt}false\end{alltt}

            \end{InvisibleVerbatim}
            
        
    
\section{Determining the Return Type of a \texttt{:call}}Here, we start with the output of
\texttt{extract\_calls\_from\_returns}.

    % Make sure that atleast 4 lines are below the HR
    \needspace{4\baselineskip}

    
        \vspace{6pt}
        \makebox[0.1\linewidth]{\smaller\hfill\tt\color{nbframe-in-prompt}In\hspace{4pt}{[}44{]}:\hspace{4pt}}\\*
        \vspace{-2.65\baselineskip}
        \begin{ColorVerbatim}
            \vspace{-0.7\baselineskip}
            \begin{Verbatim}[commandchars=\\\{\}]
\PY{n}{cs} \PY{o}{=} \PY{n}{TypeCheck}\PY{o}{.}\PY{n}{extract\PYZus{}calls\PYZus{}from\PYZus{}returns}\PY{p}{(}\PY{n}{code\PYZus{}typed}\PY{p}{(}\PY{n}{f2}\PY{p}{,}\PY{p}{(}\PY{n}{Int}\PY{p}{,}\PY{p}{)}\PY{p}{)}\PY{p}{[}\PY{l+m+mi}{1}\PY{p}{]}\PY{p}{)}
\end{Verbatim}

            
                \vspace{-0.2\baselineskip}
            
        \end{ColorVerbatim}
    

    

        % If the first block is an image, minipage the image.  Else
        % request a certain amount of space for the input text.
        \needspace{4\baselineskip}
        
        

            % Add document contents.
            
                \makebox[0.1\linewidth]{\smaller\hfill\tt\color{nbframe-out-prompt}Out\hspace{4pt}{[}44{]}:\hspace{4pt}}\\*
                \vspace{-2.55\baselineskip}\begin{InvisibleVerbatim}
                \vspace{-0.5\baselineskip}
\begin{alltt}1-element Array\{Expr,1\}:
 :(f1(top(box)(Int64,top(add\_int)(y::Int64,2))::Int64)::Union(Int64,Ma
thConst\{:π\}))\end{alltt}

            \end{InvisibleVerbatim}
            
        
    
From this, we need to extract the types of the arguments it is being
called with. (We need this to determine which method would be called.)

    % Make sure that atleast 4 lines are below the HR
    \needspace{4\baselineskip}

    
        \vspace{6pt}
        \makebox[0.1\linewidth]{\smaller\hfill\tt\color{nbframe-in-prompt}In\hspace{4pt}{[}46{]}:\hspace{4pt}}\\*
        \vspace{-2.65\baselineskip}
        \begin{ColorVerbatim}
            \vspace{-0.7\baselineskip}
            \begin{Verbatim}[commandchars=\\\{\}]
\PY{n}{cs}\PY{p}{[}\PY{l+m+mi}{1}\PY{p}{]}\PY{o}{.}\PY{n}{args}\PY{p}{[}\PY{l+m+mi}{2}\PY{p}{]}
\end{Verbatim}

            
                \vspace{-0.2\baselineskip}
            
        \end{ColorVerbatim}
    

    

        % If the first block is an image, minipage the image.  Else
        % request a certain amount of space for the input text.
        \needspace{4\baselineskip}
        
        

            % Add document contents.
            
                \makebox[0.1\linewidth]{\smaller\hfill\tt\color{nbframe-out-prompt}Out\hspace{4pt}{[}46{]}:\hspace{4pt}}\\*
                \vspace{-2.55\baselineskip}\begin{InvisibleVerbatim}
                \vspace{-0.5\baselineskip}
\begin{alltt}:(top(box)(Int64,top(add\_int)(y::Int64,2))::Int64)\end{alltt}

            \end{InvisibleVerbatim}
            
        
    
The above gets us the first argument to the \texttt{:call}, but it is
still a whole \texttt{Expr}, not a simple value. Luckily, as you can see
at the end, it has been annotated with an inferred type already.

    % Make sure that atleast 4 lines are below the HR
    \needspace{4\baselineskip}

    
        \vspace{6pt}
        \makebox[0.1\linewidth]{\smaller\hfill\tt\color{nbframe-in-prompt}In\hspace{4pt}{[}47{]}:\hspace{4pt}}\\*
        \vspace{-2.65\baselineskip}
        \begin{ColorVerbatim}
            \vspace{-0.7\baselineskip}
            \begin{Verbatim}[commandchars=\\\{\}]
\PY{n}{cs}\PY{p}{[}\PY{l+m+mi}{1}\PY{p}{]}\PY{o}{.}\PY{n}{args}\PY{p}{[}\PY{l+m+mi}{2}\PY{p}{]}\PY{o}{.}\PY{n}{args}\PY{p}{[}\PY{l+m+mi}{2}\PY{p}{]}
\end{Verbatim}

            
                \vspace{-0.2\baselineskip}
            
        \end{ColorVerbatim}
    

    

        % If the first block is an image, minipage the image.  Else
        % request a certain amount of space for the input text.
        \needspace{4\baselineskip}
        
        

            % Add document contents.
            
                \makebox[0.1\linewidth]{\smaller\hfill\tt\color{nbframe-out-prompt}Out\hspace{4pt}{[}47{]}:\hspace{4pt}}\\*
                \vspace{-2.55\baselineskip}\begin{InvisibleVerbatim}
                \vspace{-0.5\baselineskip}
\begin{alltt}:Int64\end{alltt}

            \end{InvisibleVerbatim}
            
        
    
Thus for this example, we know that it is the function named
\texttt{cs{[}1{]}.args{[}1{]}} being called with
\texttt{cs{[}1{]}.args{[}2{]}.args{[}2{]}}.

    % Make sure that atleast 4 lines are below the HR
    \needspace{4\baselineskip}

    
        \vspace{6pt}
        \makebox[0.1\linewidth]{\smaller\hfill\tt\color{nbframe-in-prompt}In\hspace{4pt}{[}48{]}:\hspace{4pt}}\\*
        \vspace{-2.65\baselineskip}
        \begin{ColorVerbatim}
            \vspace{-0.7\baselineskip}
            \begin{Verbatim}[commandchars=\\\{\}]
\PY{n}{println}\PY{p}{(}\PY{n}{cs}\PY{p}{[}\PY{l+m+mi}{1}\PY{p}{]}\PY{o}{.}\PY{n}{args}\PY{p}{[}\PY{l+m+mi}{1}\PY{p}{]}\PY{p}{)}
\PY{n}{println}\PY{p}{(}\PY{n}{cs}\PY{p}{[}\PY{l+m+mi}{1}\PY{p}{]}\PY{o}{.}\PY{n}{args}\PY{p}{[}\PY{l+m+mi}{2}\PY{p}{]}\PY{o}{.}\PY{n}{args}\PY{p}{[}\PY{l+m+mi}{2}\PY{p}{]}\PY{p}{)}
\end{Verbatim}

            
                \vspace{-0.2\baselineskip}
            
        \end{ColorVerbatim}
    

    

        % If the first block is an image, minipage the image.  Else
        % request a certain amount of space for the input text.
        \needspace{4\baselineskip}
        
        

            % Add document contents.
            
                \begin{InvisibleVerbatim}
                \vspace{-0.5\baselineskip}
\begin{alltt}f1
Int64
\end{alltt}

            \end{InvisibleVerbatim}
            
        
    
However, for this \texttt{cs{[}1{]}}, the type inference has already
provided a return type -- \texttt{Union(Int64,MathConst\{:π\})}.

    % Make sure that atleast 4 lines are below the HR
    \needspace{4\baselineskip}

    
        \vspace{6pt}
        \makebox[0.1\linewidth]{\smaller\hfill\tt\color{nbframe-in-prompt}In\hspace{4pt}{[}49{]}:\hspace{4pt}}\\*
        \vspace{-2.65\baselineskip}
        \begin{ColorVerbatim}
            \vspace{-0.7\baselineskip}
            \begin{Verbatim}[commandchars=\\\{\}]
\PY{n}{cs}\PY{p}{[}\PY{l+m+mi}{1}\PY{p}{]}\PY{o}{.}\PY{n}{typ}
\end{Verbatim}

            
                \vspace{-0.2\baselineskip}
            
        \end{ColorVerbatim}
    

    

        % If the first block is an image, minipage the image.  Else
        % request a certain amount of space for the input text.
        \needspace{4\baselineskip}
        
        

            % Add document contents.
            
                \makebox[0.1\linewidth]{\smaller\hfill\tt\color{nbframe-out-prompt}Out\hspace{4pt}{[}49{]}:\hspace{4pt}}\\*
                \vspace{-2.55\baselineskip}\begin{InvisibleVerbatim}
                \vspace{-0.5\baselineskip}
\begin{alltt}Union(Int64,MathConst\{:π\})\end{alltt}

            \end{InvisibleVerbatim}
            
        
    
Since there can be some digging around in the \texttt{Expr}s and other
\texttt{Expr}-like things (\texttt{Symbol}s,\texttt{Int}s, etc) that
occur in \texttt{:call} arugments, I wrote \texttt{find\_returntype} to
encapsulate that logic.

    % Make sure that atleast 4 lines are below the HR
    \needspace{4\baselineskip}

    
        \vspace{6pt}
        \makebox[0.1\linewidth]{\smaller\hfill\tt\color{nbframe-in-prompt}In\hspace{4pt}{[}50{]}:\hspace{4pt}}\\*
        \vspace{-2.65\baselineskip}
        \begin{ColorVerbatim}
            \vspace{-0.7\baselineskip}
            \begin{Verbatim}[commandchars=\\\{\}]
\PY{n}{function} \PY{n}{find\PYZus{}returntype}\PY{p}{(}\PY{n}{e}\PY{p}{:}\PY{p}{:}\PY{n}{Expr}\PY{p}{,}\PY{n}{context}\PY{p}{:}\PY{p}{:}\PY{n}{Expr}\PY{p}{)} \PY{c}{\PYZsh{}must be :call,:new,:call1}
  \PY{k}{if} \PY{n}{Base}\PY{o}{.}\PY{n}{is\PYZus{}expr}\PY{p}{(}\PY{n}{e}\PY{p}{,}\PY{p}{:}\PY{n}{new}\PY{p}{)}\PY{p}{;} \PY{k}{return} \PY{n}{e}\PY{o}{.}\PY{n}{typ}\PY{p}{;} \PY{n}{end}
  \PY{k}{if} \PY{n}{Base}\PY{o}{.}\PY{n}{is\PYZus{}expr}\PY{p}{(}\PY{n}{e}\PY{p}{,}\PY{p}{:}\PY{n}{call1}\PY{p}{)} \PY{o}{\PYZam{}}\PY{o}{\PYZam{}} \PY{n}{isa}\PY{p}{(}\PY{n}{e}\PY{o}{.}\PY{n}{args}\PY{p}{[}\PY{l+m+mi}{1}\PY{p}{]}\PY{p}{,} \PY{n}{TopNode}\PY{p}{)}\PY{p}{;} \PY{k}{return} \PY{n}{e}\PY{o}{.}\PY{n}{typ}\PY{p}{;} \PY{n}{end}
  \PY{k}{if} \PY{err}{!}\PY{n}{Base}\PY{o}{.}\PY{n}{is\PYZus{}expr}\PY{p}{(}\PY{n}{e}\PY{p}{,}\PY{p}{:}\PY{n}{call}\PY{p}{)}\PY{p}{;} \PY{n}{error}\PY{p}{(}\PY{l+s}{\PYZdq{}}\PY{l+s}{Expected :call Expr}\PY{l+s}{\PYZdq{}}\PY{p}{)}\PY{p}{;} \PY{n}{end}

  \PY{k}{if} \PY{n}{is\PYZus{}top}\PY{p}{(}\PY{n}{e}\PY{p}{)}
    \PY{k}{return} \PY{n}{e}\PY{o}{.}\PY{n}{typ}
  \PY{n}{end}

  \PY{n}{callee} \PY{o}{=} \PY{n}{e}\PY{o}{.}\PY{n}{args}\PY{p}{[}\PY{l+m+mi}{1}\PY{p}{]}
  \PY{k}{if} \PY{n}{is\PYZus{}top}\PY{p}{(}\PY{n}{callee}\PY{p}{)}
    \PY{k}{return} \PY{n}{find\PYZus{}returntype}\PY{p}{(}\PY{n}{callee}\PY{p}{,}\PY{n}{context}\PY{p}{)}
  \PY{n}{elseif} \PY{n}{isa}\PY{p}{(}\PY{n}{callee}\PY{p}{,}\PY{n}{SymbolNode}\PY{p}{)} \PY{c}{\PYZsh{} only seen (func::F), so non\PYZhy{}generic function}
    \PY{k}{return} \PY{n}{Any}
  \PY{n}{elseif} \PY{o+ow}{is}\PY{p}{(}\PY{n}{callee}\PY{p}{,}\PY{n}{Symbol}\PY{p}{)}
    \PY{k}{if} \PY{n}{e}\PY{o}{.}\PY{n}{typ} \PY{o}{!=} \PY{n}{Any} \PY{o}{|}\PY{o}{|} \PY{n+nb}{any}\PY{p}{(}\PY{p}{[}\PY{n}{isa}\PY{p}{(}\PY{n}{x}\PY{p}{,}\PY{n}{LambdaStaticData}\PY{p}{)} \PY{k}{for} \PY{n}{x} \PY{o+ow}{in} \PY{n}{e}\PY{o}{.}\PY{n}{args}\PY{p}{[}\PY{l+m+mi}{2}\PY{p}{:}\PY{n}{end}\PY{p}{]}\PY{p}{]}\PY{p}{)}
      \PY{k}{return} \PY{n}{e}\PY{o}{.}\PY{n}{typ}
    \PY{n}{end}

    \PY{k}{if} \PY{n}{isdefined}\PY{p}{(}\PY{n}{Base}\PY{p}{,}\PY{n}{callee}\PY{p}{)}
      \PY{n}{f} \PY{o}{=} \PY{n+nb}{eval}\PY{p}{(}\PY{n}{Base}\PY{p}{,}\PY{n}{callee}\PY{p}{)}
      \PY{k}{if} \PY{err}{!}\PY{n}{isa}\PY{p}{(}\PY{n}{f}\PY{p}{,}\PY{n}{Function}\PY{p}{)} \PY{o}{|}\PY{o}{|} \PY{err}{!}\PY{n}{isgeneric}\PY{p}{(}\PY{n}{f}\PY{p}{)}
        \PY{k}{return} \PY{n}{e}\PY{o}{.}\PY{n}{typ}
      \PY{n}{end}
      \PY{n}{fargtypes} \PY{o}{=} \PY{n+nb}{tuple}\PY{p}{(}\PY{p}{[}\PY{n}{find\PYZus{}argtype}\PY{p}{(}\PY{n}{ea}\PY{p}{,}\PY{n}{context}\PY{p}{)} \PY{k}{for} \PY{n}{ea} \PY{o+ow}{in} \PY{n}{e}\PY{o}{.}\PY{n}{args}\PY{p}{[}\PY{l+m+mi}{2}\PY{p}{:}\PY{n}{end}\PY{p}{]}\PY{p}{]}\PY{p}{)}
      \PY{k}{return} \PY{n}{Union}\PY{p}{(}\PY{p}{[}\PY{n}{returntype}\PY{p}{(}\PY{n}{ef}\PY{p}{)} \PY{k}{for} \PY{n}{ef} \PY{o+ow}{in} \PY{n}{code\PYZus{}typed}\PY{p}{(}\PY{n}{f}\PY{p}{,}\PY{n}{fargtypes}\PY{p}{)}\PY{p}{]}\PY{o}{.}\PY{o}{.}\PY{o}{.}\PY{p}{)}
    \PY{k}{else}
      \PY{k}{return} \PY{n+nd}{@show} \PY{n}{e}\PY{o}{.}\PY{n}{typ}
    \PY{n}{end}
  \PY{n}{end}

  \PY{k}{return} \PY{n}{e}\PY{o}{.}\PY{n}{typ}
\PY{n}{end}
\end{Verbatim}

            
                \vspace{-0.2\baselineskip}
            
        \end{ColorVerbatim}
    

    

        % If the first block is an image, minipage the image.  Else
        % request a certain amount of space for the input text.
        \needspace{4\baselineskip}
        
        

            % Add document contents.
            
                \makebox[0.1\linewidth]{\smaller\hfill\tt\color{nbframe-out-prompt}Out\hspace{4pt}{[}50{]}:\hspace{4pt}}\\*
                \vspace{-2.55\baselineskip}\begin{InvisibleVerbatim}
                \vspace{-0.5\baselineskip}
\begin{alltt}find\_returntype (generic function with 1 method)\end{alltt}

            \end{InvisibleVerbatim}
            
        
    
\subsection{Deciding If It's Not This Function's Fault}If a function is calling another with concrete types, there will only be
one possible method to get called. In that case, it is probably the
callee's fault that the return type is bad.

If a function is calling another with looser types, there may be
multiple possible methods that could get called. In that case, the way
the second function is being used may be causing the problem. If the
arguments to the caller are all concrete, where is it getting an
abstract type to call this other function?However, if you are just trying to see if one function is type-stable,
you might care about \texttt{foo}'s unstability, even if it's really
\texttt{barr}'s fault.\part{Stable Types Inside Loops}In Julia, for-loops are generally the fastest way to write code. (Faster
than vectorized code; faster than maps or folds.) One way to
accidentally decrease their performance is to change the type of a
variable in the loop. If all the variables in a loop have stable types,
then the code Julia outputs will be the same tight, fast code as a
typed, compiled language. If any variable has a type that changes,
slower dynamic code will be produced to handle that.It can be easy to write code that has this problem, if you're not aware
of it, even in simple programs.

    % Make sure that atleast 4 lines are below the HR
    \needspace{4\baselineskip}

    
        \vspace{6pt}
        \makebox[0.1\linewidth]{\smaller\hfill\tt\color{nbframe-in-prompt}In\hspace{4pt}{[}26{]}:\hspace{4pt}}\\*
        \vspace{-2.65\baselineskip}
        \begin{ColorVerbatim}
            \vspace{-0.7\baselineskip}
            \begin{Verbatim}[commandchars=\\\{\}]
\PY{n}{x} \PY{o}{=} \PY{l+m+mi}{5} \PY{c}{\PYZsh{} x is an Int}
\PY{k}{for} \PY{n}{i}\PY{o}{=}\PY{l+m+mi}{1}\PY{p}{:}\PY{l+m+mi}{1000}
 \PY{n}{x} \PY{o}{+}\PY{o}{=} \PY{l+m+mi}{100}
 \PY{n}{x} \PY{o}{/}\PY{o}{=} \PY{l+m+mi}{2} \PY{c}{\PYZsh{} x is a Float64}
\PY{n}{end} 
\PY{n}{x} \PY{c}{\PYZsh{}x is a Float64}
\end{Verbatim}

            
                \vspace{-0.2\baselineskip}
            
        \end{ColorVerbatim}
    

    

        % If the first block is an image, minipage the image.  Else
        % request a certain amount of space for the input text.
        \needspace{4\baselineskip}
        
        

            % Add document contents.
            
                \makebox[0.1\linewidth]{\smaller\hfill\tt\color{nbframe-out-prompt}Out\hspace{4pt}{[}26{]}:\hspace{4pt}}\\*
                \vspace{-2.55\baselineskip}\begin{InvisibleVerbatim}
                \vspace{-0.5\baselineskip}
\begin{alltt}100.0\end{alltt}

            \end{InvisibleVerbatim}
            
        
    
In this code example, \texttt{x} begins life as an \texttt{Int}. In the
first iteration of the loop, \texttt{x += 100} takes \texttt{x} as an
\texttt{Int} and returns an \texttt{Int}; \texttt{x /= 2} takes this new
\texttt{Int} and returns a \texttt{Float64}. After this, \texttt{x} will
be a \texttt{Float64} for all the remaining iterations of the loop. This
means that the extra dynamic code that is needed to handle \texttt{x}
being either an \texttt{Int} or a \texttt{Float64} slows down all the
iterations, despite only being needed for the first one. This can be
fixed by making \texttt{x} a \texttt{Float64} from the start:
\texttt{x = 5.0}.

    % Make sure that atleast 4 lines are below the HR
    \needspace{4\baselineskip}

    
        \vspace{6pt}
        \makebox[0.1\linewidth]{\smaller\hfill\tt\color{nbframe-in-prompt}In\hspace{4pt}{[}27{]}:\hspace{4pt}}\\*
        \vspace{-2.65\baselineskip}
        \begin{ColorVerbatim}
            \vspace{-0.7\baselineskip}
            \begin{Verbatim}[commandchars=\\\{\}]
\PY{n}{x} \PY{o}{=} \PY{l+m+mf}{5.0} \PY{c}{\PYZsh{} x is a Float64}
\PY{k}{for} \PY{n}{i}\PY{o}{=}\PY{l+m+mi}{1}\PY{p}{:}\PY{l+m+mi}{1000}
 \PY{n}{x} \PY{o}{+}\PY{o}{=} \PY{l+m+mi}{100}
 \PY{n}{x} \PY{o}{/}\PY{o}{=} \PY{l+m+mi}{2} \PY{c}{\PYZsh{} x is a Float64}
\PY{n}{end} 
\PY{n}{x} \PY{c}{\PYZsh{}x is a Float64}
\end{Verbatim}

            
                \vspace{-0.2\baselineskip}
            
        \end{ColorVerbatim}
    

    

        % If the first block is an image, minipage the image.  Else
        % request a certain amount of space for the input text.
        \needspace{4\baselineskip}
        
        

            % Add document contents.
            
                \makebox[0.1\linewidth]{\smaller\hfill\tt\color{nbframe-out-prompt}Out\hspace{4pt}{[}27{]}:\hspace{4pt}}\\*
                \vspace{-2.55\baselineskip}\begin{InvisibleVerbatim}
                \vspace{-0.5\baselineskip}
\begin{alltt}100.0\end{alltt}

            \end{InvisibleVerbatim}
            
        
    
Variables whose types change in loops can be detected in generic
functions by looking at the output of \texttt{code\_typed}. Since loops
are lowered to gotos, we need to first find the loops and then check the
types of the variables involved. Finding loops can be as simple as
looking for gotos that jump backwards in the function: gotos whose
labels precede them. Each instruction between the goto and its label is
part of the loop body. For each instruction in the loop body, we can
look at the inferred type of any variables involved. If the inferred
type is a UnionType (or not a leaf type), then the variable's type is
unstable.\section{Collecting the Contents of a Loop}Let's begin by defining an example function to work with.

    % Make sure that atleast 4 lines are below the HR
    \needspace{4\baselineskip}

    
        \vspace{6pt}
        \makebox[0.1\linewidth]{\smaller\hfill\tt\color{nbframe-in-prompt}In\hspace{4pt}{[}38{]}:\hspace{4pt}}\\*
        \vspace{-2.65\baselineskip}
        \begin{ColorVerbatim}
            \vspace{-0.7\baselineskip}
            \begin{Verbatim}[commandchars=\\\{\}]
\PY{n}{function} \PY{n}{bar}\PY{p}{(}\PY{n}{x}\PY{p}{:}\PY{p}{:}\PY{n}{Int}\PY{p}{)}
  \PY{k}{for} \PY{n}{i}\PY{o}{=}\PY{l+m+mi}{1}\PY{p}{:}\PY{l+m+mi}{1000}
    \PY{n}{x} \PY{o}{+}\PY{o}{=} \PY{l+m+mi}{100}
    \PY{n}{x} \PY{o}{/}\PY{o}{=} \PY{l+m+mi}{2}
  \PY{n}{end}
  \PY{n}{x}
\PY{n}{end}
\end{Verbatim}

            
                \vspace{-0.2\baselineskip}
            
        \end{ColorVerbatim}
    

    

        % If the first block is an image, minipage the image.  Else
        % request a certain amount of space for the input text.
        \needspace{4\baselineskip}
        
        

            % Add document contents.
            
                \makebox[0.1\linewidth]{\smaller\hfill\tt\color{nbframe-out-prompt}Out\hspace{4pt}{[}38{]}:\hspace{4pt}}\\*
                \vspace{-2.55\baselineskip}\begin{InvisibleVerbatim}
                \vspace{-0.5\baselineskip}
\begin{alltt}bar (generic function with 1 method)\end{alltt}

            \end{InvisibleVerbatim}
            
        
    
The goal is to write a function that takes a method of a generic
function and extracts the typed body of any loops. First, we can get the
typed body of the function.

    % Make sure that atleast 4 lines are below the HR
    \needspace{4\baselineskip}

    
        \vspace{6pt}
        \makebox[0.1\linewidth]{\smaller\hfill\tt\color{nbframe-in-prompt}In\hspace{4pt}{[}40{]}:\hspace{4pt}}\\*
        \vspace{-2.65\baselineskip}
        \begin{ColorVerbatim}
            \vspace{-0.7\baselineskip}
            \begin{Verbatim}[commandchars=\\\{\}]
\PY{n}{b} \PY{o}{=} \PY{n}{body}\PY{p}{(}\PY{n}{bar}\PY{p}{,}\PY{p}{(}\PY{n}{Int}\PY{p}{,}\PY{p}{)}\PY{p}{)}\PY{p}{[}\PY{l+m+mi}{1}\PY{p}{]}
\end{Verbatim}

            
                \vspace{-0.2\baselineskip}
            
        \end{ColorVerbatim}
    

    

        % If the first block is an image, minipage the image.  Else
        % request a certain amount of space for the input text.
        \needspace{4\baselineskip}
        
        

            % Add document contents.
            
                \makebox[0.1\linewidth]{\smaller\hfill\tt\color{nbframe-out-prompt}Out\hspace{4pt}{[}40{]}:\hspace{4pt}}\\*
                \vspace{-2.55\baselineskip}\begin{InvisibleVerbatim}
                \vspace{-0.5\baselineskip}
\begin{alltt}16-element Array\{Any,1\}:
 :( \# In[38], line 2:)
 :(\#s79 = 1)
 :(1: )
 :(unless top(sle\_int)(\#s79::Int64,1000)::Bool goto 2)
 :(i = \#s79::Int64)
 :( \# line 3:)
 :(x = +(x::Union(Int64,Float64),100)::Union(Int64,Float64))
 :( \# line 4:)
 :(x = /(x::Union(Int64,Float64),2)::Float64)
 :(3: )
 :(\#s79 = top(box)(Int64,top(add\_int)(1,\#s79::Int64))::Int64)
 :(goto 1)
 :(2: )
 :(0: )
 :( \# line 6:)
 :(return x::Union(Int64,Float64))\end{alltt}

            \end{InvisibleVerbatim}
            
        
    
The above is an array of \texttt{Expr}s and other expression types. We
want to find out if, for each \texttt{goto} and \texttt{unless}, whether
their destination label comes before them. Another way to do this is to
look at all labels, and check for \texttt{goto}s that jump back to each
one. While checking to see if we're in a loop, we'll also need to deal
with nested loops and knowing when we've finished our loop.

    % Make sure that atleast 4 lines are below the HR
    \needspace{4\baselineskip}

    
        \vspace{6pt}
        \makebox[0.1\linewidth]{\smaller\hfill\tt\color{nbframe-in-prompt}In\hspace{4pt}{[}44{]}:\hspace{4pt}}\\*
        \vspace{-2.65\baselineskip}
        \begin{ColorVerbatim}
            \vspace{-0.7\baselineskip}
            \begin{Verbatim}[commandchars=\\\{\}]
\PY{n}{inloops} \PY{o}{=} \PY{l+m+mi}{0}
\PY{n}{ends} \PY{o}{=} \PY{n}{Int}\PY{p}{[}\PY{p}{]}
\PY{n}{loopbody} \PY{o}{=} \PY{p}{\PYZob{}}\PY{p}{\PYZcb{}}
\PY{k}{for} \PY{n}{i} \PY{o+ow}{in} \PY{l+m+mi}{1}\PY{p}{:}\PY{n}{length}\PY{p}{(}\PY{n}{b}\PY{p}{)}
  \PY{k}{if} \PY{n}{typeof}\PY{p}{(}\PY{n}{b}\PY{p}{[}\PY{n}{i}\PY{p}{]}\PY{p}{)} \PY{o}{==} \PY{n}{LabelNode}
    \PY{n}{l} \PY{o}{=} \PY{n}{b}\PY{p}{[}\PY{n}{i}\PY{p}{]}\PY{o}{.}\PY{n}{label}
    \PY{n}{jumpback} \PY{o}{=} \PY{n}{findnext}\PY{p}{(}\PY{n}{x} \PY{o}{\PYZhy{}}\PY{o}{\PYZgt{}} \PY{n}{typeof}\PY{p}{(}\PY{n}{x}\PY{p}{)} \PY{o}{==} \PY{n}{GotoNode} \PY{o}{\PYZam{}}\PY{o}{\PYZam{}} \PY{n}{x}\PY{o}{.}\PY{n}{label} \PY{o}{==} \PY{n}{l}\PY{p}{,} \PY{n}{b}\PY{p}{,} \PY{n}{i}\PY{p}{)}
    \PY{k}{if} \PY{n}{jumpback} \PY{o}{!=}\PY{l+m+mi}{0} \PY{c}{\PYZsh{}then there\PYZsq{}s a goto that jumps here}
      \PY{n}{println}\PY{p}{(}\PY{l+s}{\PYZdq{}}\PY{l+s}{loop from \PYZdl{}i to \PYZdl{}jumpback}\PY{l+s}{\PYZdq{}}\PY{p}{)}
      \PY{n}{push}\PY{err}{!}\PY{p}{(}\PY{n}{ends}\PY{p}{,}\PY{n}{jumpback}\PY{p}{)}
      \PY{n}{inloops} \PY{o}{+}\PY{o}{=} \PY{l+m+mi}{1}
    \PY{n}{end}
  \PY{n}{end}
 
  \PY{k}{if} \PY{n}{inloops} \PY{o}{\PYZgt{}} \PY{l+m+mi}{0}
    \PY{n}{println}\PY{p}{(}\PY{l+s}{\PYZdq{}}\PY{l+s+se}{\PYZbs{}t}\PY{l+s}{\PYZdl{}(b[i])}\PY{l+s}{\PYZdq{}}\PY{p}{)}
    \PY{n}{push}\PY{err}{!}\PY{p}{(}\PY{n}{loopbody}\PY{p}{,}\PY{n}{b}\PY{p}{[}\PY{n}{i}\PY{p}{]}\PY{p}{)}
  \PY{n}{end}

  \PY{k}{if} \PY{n}{i} \PY{o+ow}{in} \PY{n}{ends}
    \PY{n}{splice}\PY{err}{!}\PY{p}{(}\PY{n}{ends}\PY{p}{,}\PY{n}{findfirst}\PY{p}{(}\PY{n}{ends}\PY{p}{,}\PY{n}{i}\PY{p}{)}\PY{p}{)}
    \PY{n}{inloops} \PY{o}{\PYZhy{}}\PY{o}{=} \PY{l+m+mi}{1}
  \PY{n}{end}
 
\PY{n}{end}
\end{Verbatim}

            
                \vspace{-0.2\baselineskip}
            
        \end{ColorVerbatim}
    

    

        % If the first block is an image, minipage the image.  Else
        % request a certain amount of space for the input text.
        \needspace{4\baselineskip}
        
        

            % Add document contents.
            
                \begin{InvisibleVerbatim}
                \vspace{-0.5\baselineskip}
\begin{alltt}loop from 3 to 12
        :(1: )
        :(unless top(sle\_int)(\#s79::Int64,1000)::Bool goto 2)
        :(i = \#s79::Int64)
        :( \# line 3:)
        :(x = +(x::Union(Int64,Float64),100)::Union(Int64,Float64))
        :( \# line 4:)
        :(x = /(x::Union(Int64,Float64),2)::Float64)
        :(3: )
        :(\#s79 = top(box)(Int64,top(add\_int)(1,\#s79::Int64))::Int64)
        :(goto 1)
\end{alltt}

            \end{InvisibleVerbatim}
            
        
    
We still have a lot of noise here, in the form of line numbers and
irrelevant labels.

    % Make sure that atleast 4 lines are below the HR
    \needspace{4\baselineskip}

    
        \vspace{6pt}
        \makebox[0.1\linewidth]{\smaller\hfill\tt\color{nbframe-in-prompt}In\hspace{4pt}{[}31{]}:\hspace{4pt}}\\*
        \vspace{-2.65\baselineskip}
        \begin{ColorVerbatim}
            \vspace{-0.7\baselineskip}
            \begin{Verbatim}[commandchars=\\\{\}]
\PY{n}{typeof}\PY{p}{(}\PY{n}{loopbody}\PY{p}{[}\PY{l+m+mi}{1}\PY{p}{]}\PY{p}{)} \PY{c}{\PYZsh{}:(1: )}
\end{Verbatim}

            
                \vspace{-0.2\baselineskip}
            
        \end{ColorVerbatim}
    

    

        % If the first block is an image, minipage the image.  Else
        % request a certain amount of space for the input text.
        \needspace{4\baselineskip}
        
        

            % Add document contents.
            
                \makebox[0.1\linewidth]{\smaller\hfill\tt\color{nbframe-out-prompt}Out\hspace{4pt}{[}31{]}:\hspace{4pt}}\\*
                \vspace{-2.55\baselineskip}\begin{InvisibleVerbatim}
                \vspace{-0.5\baselineskip}
\begin{alltt}LabelNode\end{alltt}

            \end{InvisibleVerbatim}
            
        
    


    % Make sure that atleast 4 lines are below the HR
    \needspace{4\baselineskip}

    
        \vspace{6pt}
        \makebox[0.1\linewidth]{\smaller\hfill\tt\color{nbframe-in-prompt}In\hspace{4pt}{[}32{]}:\hspace{4pt}}\\*
        \vspace{-2.65\baselineskip}
        \begin{ColorVerbatim}
            \vspace{-0.7\baselineskip}
            \begin{Verbatim}[commandchars=\\\{\}]
\PY{n}{typeof}\PY{p}{(}\PY{n}{loopbody}\PY{p}{[}\PY{l+m+mi}{4}\PY{p}{]}\PY{p}{)} \PY{c}{\PYZsh{}:( \PYZsh{} line 3:)}
\end{Verbatim}

            
                \vspace{-0.2\baselineskip}
            
        \end{ColorVerbatim}
    

    

        % If the first block is an image, minipage the image.  Else
        % request a certain amount of space for the input text.
        \needspace{4\baselineskip}
        
        

            % Add document contents.
            
                \makebox[0.1\linewidth]{\smaller\hfill\tt\color{nbframe-out-prompt}Out\hspace{4pt}{[}32{]}:\hspace{4pt}}\\*
                \vspace{-2.55\baselineskip}\begin{InvisibleVerbatim}
                \vspace{-0.5\baselineskip}
\begin{alltt}LineNumberNode\end{alltt}

            \end{InvisibleVerbatim}
            
        
    


    % Make sure that atleast 4 lines are below the HR
    \needspace{4\baselineskip}

    
        \vspace{6pt}
        \makebox[0.1\linewidth]{\smaller\hfill\tt\color{nbframe-in-prompt}In\hspace{4pt}{[}45{]}:\hspace{4pt}}\\*
        \vspace{-2.65\baselineskip}
        \begin{ColorVerbatim}
            \vspace{-0.7\baselineskip}
            \begin{Verbatim}[commandchars=\\\{\}]
\PY{n+nb}{filter}\PY{p}{(}\PY{n}{x} \PY{o}{\PYZhy{}}\PY{o}{\PYZgt{}} \PY{n}{typeof}\PY{p}{(}\PY{n}{x}\PY{p}{)} \PY{o}{!=} \PY{n}{LineNumberNode} \PY{o}{\PYZam{}}\PY{o}{\PYZam{}} \PY{n}{typeof}\PY{p}{(}\PY{n}{x}\PY{p}{)} \PY{o}{!=} \PY{n}{LabelNode}\PY{p}{,} \PY{n}{loopbody}\PY{p}{)}
\end{Verbatim}

            
                \vspace{-0.2\baselineskip}
            
        \end{ColorVerbatim}
    

    

        % If the first block is an image, minipage the image.  Else
        % request a certain amount of space for the input text.
        \needspace{4\baselineskip}
        
        

            % Add document contents.
            
                \makebox[0.1\linewidth]{\smaller\hfill\tt\color{nbframe-out-prompt}Out\hspace{4pt}{[}45{]}:\hspace{4pt}}\\*
                \vspace{-2.55\baselineskip}\begin{InvisibleVerbatim}
                \vspace{-0.5\baselineskip}
\begin{alltt}6-element Array\{Any,1\}:
 :(unless top(sle\_int)(\#s79::Int64,1000)::Bool goto 2)
 :(i = \#s79::Int64)
 :(x = +(x::Union(Int64,Float64),100)::Union(Int64,Float64))
 :(x = /(x::Union(Int64,Float64),2)::Float64)
 :(\#s79 = top(box)(Int64,top(add\_int)(1,\#s79::Int64))::Int64)
 :(goto 1)\end{alltt}

            \end{InvisibleVerbatim}
            
        
    
This is quite nice. We have the \texttt{unless} that contains the exit
condition for the for loop, we have the loop variable \texttt{i} and
\texttt{\#s79}, and we have the two function calls that modify
\texttt{x}.

    % Make sure that atleast 4 lines are below the HR
    \needspace{4\baselineskip}

    
        \vspace{6pt}
        \makebox[0.1\linewidth]{\smaller\hfill\tt\color{nbframe-in-prompt}In\hspace{4pt}{[}34{]}:\hspace{4pt}}\\*
        \vspace{-2.65\baselineskip}
        \begin{ColorVerbatim}
            \vspace{-0.7\baselineskip}
            \begin{Verbatim}[commandchars=\\\{\}]
  \PY{c}{\PYZsh{} This is a function for trying to detect loops in a method of a generic function}
  \PY{c}{\PYZsh{} It takes the same arguments as code\PYZus{}typed}
  \PY{c}{\PYZsh{} And returns the lines that are inside one or more loops}
  \PY{n}{function} \PY{n}{loopcontents}\PY{p}{(}\PY{n}{args}\PY{o}{.}\PY{o}{.}\PY{o}{.}\PY{p}{)}
    \PY{n}{e} \PY{o}{=} \PY{n}{code\PYZus{}typed}\PY{p}{(}\PY{n}{args}\PY{o}{.}\PY{o}{.}\PY{o}{.}\PY{p}{)}\PY{p}{[}\PY{l+m+mi}{1}\PY{p}{]}
    \PY{n}{body} \PY{o}{=} \PY{n}{e}\PY{o}{.}\PY{n}{args}\PY{p}{[}\PY{l+m+mi}{3}\PY{p}{]}\PY{o}{.}\PY{n}{args}
    \PY{n}{loops} \PY{o}{=} \PY{n}{Int}\PY{p}{[}\PY{p}{]}
    \PY{n}{nesting} \PY{o}{=} \PY{l+m+mi}{0}
    \PY{n}{lines} \PY{o}{=} \PY{p}{\PYZob{}}\PY{p}{\PYZcb{}}
    \PY{k}{for} \PY{n}{i} \PY{o+ow}{in} \PY{l+m+mi}{1}\PY{p}{:}\PY{n}{length}\PY{p}{(}\PY{n}{body}\PY{p}{)}
      \PY{k}{if} \PY{n}{typeof}\PY{p}{(}\PY{n}{body}\PY{p}{[}\PY{n}{i}\PY{p}{]}\PY{p}{)} \PY{o}{==} \PY{n}{LabelNode}
        \PY{n}{l} \PY{o}{=} \PY{n}{body}\PY{p}{[}\PY{n}{i}\PY{p}{]}\PY{o}{.}\PY{n}{label}
        \PY{n}{jumpback} \PY{o}{=} \PY{n}{findnext}\PY{p}{(}\PY{n}{x}\PY{o}{\PYZhy{}}\PY{o}{\PYZgt{}} \PY{n}{typeof}\PY{p}{(}\PY{n}{x}\PY{p}{)} \PY{o}{==} \PY{n}{GotoNode} \PY{o}{\PYZam{}}\PY{o}{\PYZam{}} \PY{n}{x}\PY{o}{.}\PY{n}{label} \PY{o}{==} \PY{n}{l}\PY{p}{,} \PY{n}{body}\PY{p}{,} \PY{n}{i}\PY{p}{)}
        \PY{k}{if} \PY{n}{jumpback} \PY{o}{!=} \PY{l+m+mi}{0}
          \PY{c}{\PYZsh{}println(\PYZdq{}\PYZdl{}i: START LOOP: ends at \PYZdl{}jumpback\PYZdq{})}
          \PY{n}{push}\PY{err}{!}\PY{p}{(}\PY{n}{loops}\PY{p}{,}\PY{n}{jumpback}\PY{p}{)}
          \PY{n}{nesting} \PY{o}{+}\PY{o}{=} \PY{l+m+mi}{1}
        \PY{n}{end}
      \PY{n}{end}

      \PY{k}{if} \PY{n}{nesting} \PY{o}{\PYZgt{}} \PY{l+m+mi}{0}
        \PY{c}{\PYZsh{}if typeof(body[i]) == Expr}
        \PY{c}{\PYZsh{}  println(\PYZdq{}\PYZdl{}i: \PYZbs{}t\PYZdq{}, body[i])}
        \PY{c}{\PYZsh{}elseif typeof(body[i]) == LabelNode || typeof(body[i]) == GotoNode}
        \PY{c}{\PYZsh{}  println(\PYZdq{}\PYZdl{}i: \PYZdq{}, typeof(body[i]), \PYZdq{} \PYZdq{}, body[i].label)}
        \PY{c}{\PYZsh{}elseif typeof(body[i]) != LineNumberNode}
        \PY{c}{\PYZsh{}  println(\PYZdq{}\PYZdl{}i: \PYZdq{}, typeof(body[i]))}
        \PY{c}{\PYZsh{}end}
        \PY{n}{push}\PY{err}{!}\PY{p}{(}\PY{n}{lines}\PY{p}{,}\PY{p}{(}\PY{n}{i}\PY{p}{,}\PY{n}{body}\PY{p}{[}\PY{n}{i}\PY{p}{]}\PY{p}{)}\PY{p}{)}
      \PY{n}{end}

      \PY{k}{if} \PY{n}{typeof}\PY{p}{(}\PY{n}{body}\PY{p}{[}\PY{n}{i}\PY{p}{]}\PY{p}{)} \PY{o}{==} \PY{n}{GotoNode} \PY{o}{\PYZam{}}\PY{o}{\PYZam{}} \PY{o+ow}{in}\PY{p}{(}\PY{n}{i}\PY{p}{,}\PY{n}{loops}\PY{p}{)}
        \PY{n}{splice}\PY{err}{!}\PY{p}{(}\PY{n}{loops}\PY{p}{,}\PY{n}{findfirst}\PY{p}{(}\PY{n}{loops}\PY{p}{,}\PY{n}{i}\PY{p}{)}\PY{p}{)}
        \PY{n}{nesting} \PY{o}{\PYZhy{}}\PY{o}{=} \PY{l+m+mi}{1}
        \PY{c}{\PYZsh{}println(\PYZdq{}\PYZdl{}i: END LOOP: jumps to \PYZdq{},body[i].label)}
      \PY{n}{end}
    \PY{n}{end}
    \PY{n}{lines}
  \PY{n}{end}
\end{Verbatim}

            
                \vspace{-0.2\baselineskip}
            
        \end{ColorVerbatim}
    

    

        % If the first block is an image, minipage the image.  Else
        % request a certain amount of space for the input text.
        \needspace{4\baselineskip}
        
        

            % Add document contents.
            
                \makebox[0.1\linewidth]{\smaller\hfill\tt\color{nbframe-out-prompt}Out\hspace{4pt}{[}34{]}:\hspace{4pt}}\\*
                \vspace{-2.55\baselineskip}\begin{InvisibleVerbatim}
                \vspace{-0.5\baselineskip}
\begin{alltt}loopcontents (generic function with 1 method)\end{alltt}

            \end{InvisibleVerbatim}
            
        
    


    % Make sure that atleast 4 lines are below the HR
    \needspace{4\baselineskip}

    
        \vspace{6pt}
        \makebox[0.1\linewidth]{\smaller\hfill\tt\color{nbframe-in-prompt}In\hspace{4pt}{[}35{]}:\hspace{4pt}}\\*
        \vspace{-2.65\baselineskip}
        \begin{ColorVerbatim}
            \vspace{-0.7\baselineskip}
            \begin{Verbatim}[commandchars=\\\{\}]
  \PY{n}{function} \PY{n}{find\PYZus{}loose\PYZus{}types}\PY{p}{(}\PY{n}{arr}\PY{p}{:}\PY{p}{:}\PY{n}{Vector}\PY{p}{)}
    \PY{n}{lines} \PY{o}{=} \PY{n}{ASCIIString}\PY{p}{[}\PY{p}{]}
    \PY{k}{for} \PY{p}{(}\PY{n}{i}\PY{p}{,}\PY{n}{e}\PY{p}{)} \PY{o+ow}{in} \PY{n}{arr}
      \PY{k}{if} \PY{n}{typeof}\PY{p}{(}\PY{n}{e}\PY{p}{)} \PY{o}{==} \PY{n}{Expr}
        \PY{n}{es} \PY{o}{=} \PY{n}{copy}\PY{p}{(}\PY{n}{e}\PY{o}{.}\PY{n}{args}\PY{p}{)}
        \PY{k}{while} \PY{err}{!}\PY{n}{isempty}\PY{p}{(}\PY{n}{es}\PY{p}{)}
          \PY{n}{e1} \PY{o}{=} \PY{n}{pop}\PY{err}{!}\PY{p}{(}\PY{n}{es}\PY{p}{)}
          \PY{k}{if} \PY{n}{typeof}\PY{p}{(}\PY{n}{e1}\PY{p}{)} \PY{o}{==} \PY{n}{Expr}
            \PY{n}{append}\PY{err}{!}\PY{p}{(}\PY{n}{es}\PY{p}{,}\PY{n}{e1}\PY{o}{.}\PY{n}{args}\PY{p}{)}
          \PY{n}{elseif} \PY{n}{typeof}\PY{p}{(}\PY{n}{e1}\PY{p}{)} \PY{o}{==} \PY{n}{SymbolNode} \PY{o}{\PYZam{}}\PY{o}{\PYZam{}} \PY{err}{!}\PY{n}{isleaftype}\PY{p}{(}\PY{n}{e1}\PY{o}{.}\PY{n}{typ}\PY{p}{)} \PY{o}{\PYZam{}}\PY{o}{\PYZam{}} \PY{n}{typeof}\PY{p}{(}\PY{n}{e1}\PY{o}{.}\PY{n}{typ}\PY{p}{)} \PY{o}{==} \PY{n}{UnionType}
            \PY{n}{push}\PY{err}{!}\PY{p}{(}\PY{n}{lines}\PY{p}{,}\PY{l+s}{\PYZdq{}}\PY{l+s+se}{\PYZbs{}t}\PY{l+s+se}{\PYZbs{}t}\PY{l+s}{\PYZdl{}i: \PYZdl{}(e1.name): \PYZdl{}(e1.typ)}\PY{l+s}{\PYZdq{}}\PY{p}{)}
          \PY{n}{end} 
        \PY{n}{end}                          
      \PY{n}{end}
    \PY{n}{end}
    \PY{n}{isempty}\PY{p}{(}\PY{n}{lines}\PY{p}{)} \PY{err}{?} \PY{n}{lines} \PY{p}{:} \PY{n}{unshift}\PY{err}{!}\PY{p}{(}\PY{n}{lines}\PY{p}{,}\PY{l+s}{\PYZdq{}}\PY{l+s}{\PYZdq{}}\PY{p}{)}
  \PY{n}{end}
\end{Verbatim}

            
                \vspace{-0.2\baselineskip}
            
        \end{ColorVerbatim}
    

    

        % If the first block is an image, minipage the image.  Else
        % request a certain amount of space for the input text.
        \needspace{4\baselineskip}
        
        

            % Add document contents.
            
                \makebox[0.1\linewidth]{\smaller\hfill\tt\color{nbframe-out-prompt}Out\hspace{4pt}{[}35{]}:\hspace{4pt}}\\*
                \vspace{-2.55\baselineskip}\begin{InvisibleVerbatim}
                \vspace{-0.5\baselineskip}
\begin{alltt}find\_loose\_types (generic function with 1 method)\end{alltt}

            \end{InvisibleVerbatim}
            
        
    
\part{Statically Detecting \texttt{NoMethodError}s}The most obviously useful kind of type checking in Julia is to prevent
``No Method Error''s. This presents a challenge: methods are often added
to generic functions after the fact in Julia, so this checking must be
done with as much awareness as possible of the environment in which the
call will be made in order to avoid false positives.I have not yet implemented this because I don't know how to get into the
proper context to make the check. I could write first version which
ignores this problem.

    % Make sure that atleast 4 lines are below the HR
    \needspace{4\baselineskip}

    
        \vspace{6pt}
        \makebox[0.1\linewidth]{\smaller\hfill\tt\color{nbframe-in-prompt}In\hspace{4pt}{[}36{]}:\hspace{4pt}}\\*
        \vspace{-2.65\baselineskip}
        \begin{ColorVerbatim}
            \vspace{-0.7\baselineskip}
            \begin{Verbatim}[commandchars=\\\{\}]
\PY{n+nb}{type} \PY{n}{CallSignature}
  \PY{n}{name}\PY{p}{:}\PY{p}{:}\PY{n}{Symbol}
  \PY{n}{argtypes}\PY{p}{:}\PY{p}{:}\PY{n}{Array}\PY{p}{\PYZob{}}\PY{n}{DataType}\PY{p}{,}\PY{l+m+mi}{1}\PY{p}{\PYZcb{}}
\PY{n}{end}
\end{Verbatim}

            
                \vspace{-0.2\baselineskip}
            
        \end{ColorVerbatim}
    


    % Make sure that atleast 4 lines are below the HR
    \needspace{4\baselineskip}

    
        \vspace{6pt}
        \makebox[0.1\linewidth]{\smaller\hfill\tt\color{nbframe-in-prompt}In\hspace{4pt}{[}37{]}:\hspace{4pt}}\\*
        \vspace{-2.65\baselineskip}
        \begin{ColorVerbatim}
            \vspace{-0.7\baselineskip}
            \begin{Verbatim}[commandchars=\\\{\}]
\PY{n}{function} \PY{n}{find\PYZus{}no\PYZus{}method\PYZus{}errors}\PY{p}{(}\PY{n}{args}\PY{o}{.}\PY{o}{.}\PY{o}{.}\PY{p}{;}\PY{n}{mod}\PY{o}{=}\PY{n+nb+bp}{None}\PY{p}{)}
  \PY{n}{callsigs} \PY{o}{=} \PY{n}{find\PYZus{}method\PYZus{}calls}\PY{p}{(}\PY{n}{args}\PY{o}{.}\PY{o}{.}\PY{o}{.}\PY{p}{)}
  \PY{n}{output} \PY{o}{=} \PY{p}{(}\PY{n}{Function}\PY{p}{,}\PY{n}{CallSignature}\PY{p}{)}\PY{p}{[}\PY{p}{]}
  \PY{k}{for} \PY{n}{callsig} \PY{o+ow}{in} \PY{n}{callsigs}
    \PY{n}{f} \PY{o}{=} \PY{n}{mod} \PY{o}{==} \PY{n+nb+bp}{None} \PY{err}{?} \PY{n+nb}{eval}\PY{p}{(}\PY{n}{callsig}\PY{o}{.}\PY{n}{name}\PY{p}{)} \PY{p}{:} \PY{n+nb}{eval}\PY{p}{(}\PY{n}{mod}\PY{p}{,}\PY{n}{callsig}\PY{o}{.}\PY{n}{name}\PY{p}{)}
    \PY{n}{options} \PY{o}{=} \PY{n}{methods}\PY{p}{(}\PY{n}{f}\PY{p}{,}\PY{n+nb}{tuple}\PY{p}{(}\PY{n}{callsig}\PY{o}{.}\PY{n}{argtypes}\PY{o}{.}\PY{o}{.}\PY{o}{.}\PY{p}{)}\PY{p}{)}
    \PY{k}{if} \PY{n}{length}\PY{p}{(}\PY{n}{options}\PY{p}{)} \PY{o}{==} \PY{l+m+mi}{0}
      \PY{n}{push}\PY{err}{!}\PY{p}{(}\PY{n}{output}\PY{p}{,}\PY{p}{(}\PY{n}{f}\PY{p}{,}\PY{n}{callsig}\PY{p}{)}\PY{p}{)}
    \PY{n}{end}
  \PY{n}{end}
  \PY{n}{output}
\PY{n}{end}
\end{Verbatim}

            
                \vspace{-0.2\baselineskip}
            
        \end{ColorVerbatim}
    

    

        % If the first block is an image, minipage the image.  Else
        % request a certain amount of space for the input text.
        \needspace{4\baselineskip}
        
        

            % Add document contents.
            
                \makebox[0.1\linewidth]{\smaller\hfill\tt\color{nbframe-out-prompt}Out\hspace{4pt}{[}37{]}:\hspace{4pt}}\\*
                \vspace{-2.55\baselineskip}\begin{InvisibleVerbatim}
                \vspace{-0.5\baselineskip}
\begin{alltt}find\_no\_method\_errors (generic function with 1 method)\end{alltt}

            \end{InvisibleVerbatim}
            
        
    


    % Make sure that atleast 4 lines are below the HR
    \needspace{4\baselineskip}

    
        \vspace{6pt}
        \makebox[0.1\linewidth]{\smaller\hfill\tt\color{nbframe-in-prompt}In\hspace{4pt}{[}38{]}:\hspace{4pt}}\\*
        \vspace{-2.65\baselineskip}
        \begin{ColorVerbatim}
            \vspace{-0.7\baselineskip}
            \begin{Verbatim}[commandchars=\\\{\}]
\PY{n}{function} \PY{n}{find\PYZus{}method\PYZus{}calls}\PY{p}{(}\PY{n}{args}\PY{o}{.}\PY{o}{.}\PY{o}{.}\PY{p}{)}
  \PY{n}{e} \PY{o}{=} \PY{n}{code\PYZus{}typed}\PY{p}{(}\PY{n}{args}\PY{o}{.}\PY{o}{.}\PY{o}{.}\PY{p}{)}\PY{p}{[}\PY{l+m+mi}{1}\PY{p}{]}
  \PY{n}{body} \PY{o}{=} \PY{n}{e}\PY{o}{.}\PY{n}{args}\PY{p}{[}\PY{l+m+mi}{3}\PY{p}{]}\PY{o}{.}\PY{n}{args}
  \PY{n}{lines} \PY{o}{=} \PY{n}{CallSignature}\PY{p}{[}\PY{p}{]}
  \PY{k}{for} \PY{n}{b} \PY{o+ow}{in} \PY{n}{body}
    \PY{k}{if} \PY{n}{typeof}\PY{p}{(}\PY{n}{b}\PY{p}{)} \PY{o}{==} \PY{n}{Expr}
      \PY{k}{if} \PY{n}{b}\PY{o}{.}\PY{n}{head} \PY{o}{==} \PY{p}{:}\PY{k}{return} \PY{c}{\PYZsh{} want to catch function calls nested in a return}
        \PY{n}{append}\PY{err}{!}\PY{p}{(}\PY{n}{body}\PY{p}{,}\PY{n}{b}\PY{o}{.}\PY{n}{args}\PY{p}{)}
      \PY{n}{elseif} \PY{n}{b}\PY{o}{.}\PY{n}{head} \PY{o}{==} \PY{p}{:}\PY{n}{call}
        \PY{k}{if} \PY{n}{typeof}\PY{p}{(}\PY{n}{b}\PY{o}{.}\PY{n}{args}\PY{p}{[}\PY{l+m+mi}{1}\PY{p}{]}\PY{p}{)} \PY{o}{==} \PY{n}{Symbol}
          \PY{c}{\PYZsh{}@show b.args, typeof(b.args[3])}
          \PY{n}{cs} \PY{o}{=} \PY{n}{CallSignature}\PY{p}{(}\PY{n}{b}\PY{o}{.}\PY{n}{args}\PY{p}{[}\PY{l+m+mi}{1}\PY{p}{]}\PY{p}{,}\PY{p}{[}\PY{n}{typeof}\PY{p}{(}\PY{n}{e}\PY{p}{)} \PY{o}{==} \PY{n}{Expr} \PY{err}{?} \PY{n}{e}\PY{o}{.}\PY{n}{typ} \PY{p}{:}
            \PY{n}{typeof}\PY{p}{(}\PY{n}{e}\PY{p}{)} \PY{o}{==} \PY{n}{Symbol} \PY{err}{?} \PY{n}{Any} \PY{p}{:}
            \PY{n}{typeof}\PY{p}{(}\PY{n}{e}\PY{p}{)} \PY{k}{for} \PY{n}{e} \PY{o+ow}{in} \PY{n}{b}\PY{o}{.}\PY{n}{args}\PY{p}{[}\PY{l+m+mi}{2}\PY{p}{:}\PY{p}{]}\PY{p}{]}\PY{p}{)}
          \PY{n}{push}\PY{err}{!}\PY{p}{(}\PY{n}{lines}\PY{p}{,}\PY{n}{cs}\PY{p}{)}
        \PY{n}{end}
      \PY{n}{end}
    \PY{n}{end}
  \PY{n}{end}
  \PY{n}{lines}
\PY{n}{end}
\end{Verbatim}

            
                \vspace{-0.2\baselineskip}
            
        \end{ColorVerbatim}
    

    

        % If the first block is an image, minipage the image.  Else
        % request a certain amount of space for the input text.
        \needspace{4\baselineskip}
        
        

            % Add document contents.
            
                \makebox[0.1\linewidth]{\smaller\hfill\tt\color{nbframe-out-prompt}Out\hspace{4pt}{[}38{]}:\hspace{4pt}}\\*
                \vspace{-2.55\baselineskip}\begin{InvisibleVerbatim}
                \vspace{-0.5\baselineskip}
\begin{alltt}find\_method\_calls (generic function with 1 method)\end{alltt}

            \end{InvisibleVerbatim}
            
        
    


    % Make sure that atleast 4 lines are below the HR
    \needspace{4\baselineskip}

    
        \vspace{6pt}
        \makebox[0.1\linewidth]{\smaller\hfill\tt\color{nbframe-in-prompt}In\hspace{4pt}{[}39{]}:\hspace{4pt}}\\*
        \vspace{-2.65\baselineskip}
        \begin{ColorVerbatim}
            \vspace{-0.7\baselineskip}
            \begin{Verbatim}[commandchars=\\\{\}]
\PY{n}{function} \PY{n}{aba}\PY{p}{(}\PY{n}{x}\PY{p}{)}
  \PY{l+m+mi}{2}
\PY{n}{end}
\PY{n}{function} \PY{n}{aba}\PY{p}{(}\PY{n}{x}\PY{p}{,}\PY{n}{y}\PY{p}{:}\PY{p}{:}\PY{n}{Float64}\PY{p}{)}
  \PY{l+m+mf}{3.14}
\PY{n}{end}
\PY{n}{function} \PY{n}{foo}\PY{p}{(}\PY{n}{x}\PY{p}{)}
  \PY{n}{aba}\PY{p}{(}\PY{n}{x}\PY{p}{,}\PY{l+m+mi}{4}\PY{p}{)}
\PY{n}{end}
\end{Verbatim}

            
                \vspace{-0.2\baselineskip}
            
        \end{ColorVerbatim}
    

    

        % If the first block is an image, minipage the image.  Else
        % request a certain amount of space for the input text.
        \needspace{4\baselineskip}
        
        

            % Add document contents.
            
                \makebox[0.1\linewidth]{\smaller\hfill\tt\color{nbframe-out-prompt}Out\hspace{4pt}{[}39{]}:\hspace{4pt}}\\*
                \vspace{-2.55\baselineskip}\begin{InvisibleVerbatim}
                \vspace{-0.5\baselineskip}
\begin{alltt}foo (generic function with 2 methods)\end{alltt}

            \end{InvisibleVerbatim}
            
        
    


    % Make sure that atleast 4 lines are below the HR
    \needspace{4\baselineskip}

    
        \vspace{6pt}
        \makebox[0.1\linewidth]{\smaller\hfill\tt\color{nbframe-in-prompt}In\hspace{4pt}{[}42{]}:\hspace{4pt}}\\*
        \vspace{-2.65\baselineskip}
        \begin{ColorVerbatim}
            \vspace{-0.7\baselineskip}
            \begin{Verbatim}[commandchars=\\\{\}]
\PY{n}{c} \PY{o}{=} \PY{n}{find\PYZus{}no\PYZus{}method\PYZus{}errors}\PY{p}{(}\PY{n}{foo}\PY{p}{,}\PY{p}{(}\PY{n}{Float64}\PY{p}{,}\PY{p}{)}\PY{p}{)}
\end{Verbatim}

            
                \vspace{-0.2\baselineskip}
            
        \end{ColorVerbatim}
    

    

        % If the first block is an image, minipage the image.  Else
        % request a certain amount of space for the input text.
        \needspace{4\baselineskip}
        
        

            % Add document contents.
            
                \makebox[0.1\linewidth]{\smaller\hfill\tt\color{nbframe-out-prompt}Out\hspace{4pt}{[}42{]}:\hspace{4pt}}\\*
                \vspace{-2.55\baselineskip}\begin{InvisibleVerbatim}
                \vspace{-0.5\baselineskip}
\begin{alltt}1-element Array\{(Function,CallSignature),1\}:
 (aba,CallSignature(:aba,[SymbolNode,Int64]))\end{alltt}

            \end{InvisibleVerbatim}
            
        
    


    % Make sure that atleast 4 lines are below the HR
    \needspace{4\baselineskip}

    
        \vspace{6pt}
        \makebox[0.1\linewidth]{\smaller\hfill\tt\color{nbframe-in-prompt}In\hspace{4pt}{[}44{]}:\hspace{4pt}}\\*
        \vspace{-2.65\baselineskip}
        \begin{ColorVerbatim}
            \vspace{-0.7\baselineskip}
            \begin{Verbatim}[commandchars=\\\{\}]
\PY{n}{code\PYZus{}typed}\PY{p}{(}\PY{n}{foo}\PY{p}{,}\PY{p}{(}\PY{n}{Number}\PY{p}{,}\PY{p}{)}\PY{p}{)}
\end{Verbatim}

            
                \vspace{-0.2\baselineskip}
            
        \end{ColorVerbatim}
    

    

        % If the first block is an image, minipage the image.  Else
        % request a certain amount of space for the input text.
        \needspace{4\baselineskip}
        
        

            % Add document contents.
            
                \makebox[0.1\linewidth]{\smaller\hfill\tt\color{nbframe-out-prompt}Out\hspace{4pt}{[}44{]}:\hspace{4pt}}\\*
                \vspace{-2.55\baselineskip}\begin{InvisibleVerbatim}
                \vspace{-0.5\baselineskip}
\begin{alltt}2-element Array\{Any,1\}:
 :(\$(Expr(:lambda, \{:x\}, \{\{\},\{\{:x,Int64,0\}\},\{\}\}, quote  \# In[12], line
2:
        unless top(slt\_int)(5,x::Int64)::Bool goto 0 \# line 3:
        return x::Int64
        goto 1
        0:  \# In[12], line 5:
        return false
        1:
    end)))
 :(\$(Expr(:lambda, \{:x\}, \{\{\},\{\{:x,Number,0\}\},\{\}\}, quote  \# In[39],
line 8:
        return aba(x::Number,4)::None
    end)))\end{alltt}

            \end{InvisibleVerbatim}
            
        
    


    % Make sure that atleast 4 lines are below the HR
    \needspace{4\baselineskip}

    
        \vspace{6pt}
        \makebox[0.1\linewidth]{\smaller\hfill\tt\color{nbframe-in-prompt}In\hspace{4pt}{[}{]}:\hspace{4pt}}\\*
        \vspace{-2.65\baselineskip}
        \begin{ColorVerbatim}
            \vspace{-0.7\baselineskip}
            \begin{Verbatim}[commandchars=\\\{\}]

\end{Verbatim}

            
                \vspace{0.3\baselineskip}
            
        \end{ColorVerbatim}
    

        

        \renewcommand{\indexname}{Index}
        \printindex

    % End of document
    \end{document}


